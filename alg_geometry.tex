
\section{Algebraic geometry}
In this section we define the fundamental objects in algebraic geometry and state
some facts about their structure. We will then move on to the theory of
curves and Weil divisors.

\begin{mydef}
\emph{Projective n-space} over a field $k$ denoted $\mathbb{P}^n$ is the set 
of all $(n+1)$-tuples $$(x_0,\ldots,x_n)\in\mathbb{A}^{n+1}$$
modulo the equivalence relation given by $(x_0,\ldots,x_n)\sim(y_0,\ldots,y_n)$ 
if there exists $\lambda\in k$ such that $x_i=\lambda y_i$.
The equivalence class $\{(x_0,\ldots,x_n)\}$ is denoted $[x_0,\ldots,x_n]$.
Here $\mathbb{A}^n = \{ (x_1,\ldots,x_n) : x_i \in \bar{k} \} $ is the affine $n$-space.
\end{mydef}

Let $Gal(\bar{k}/k)$ be the galois group of $\bar{k}/k$. This group acts on
$\mathbb{A}^n$, such that when $\sigma \in Gal(\bar{k}/k)$ and $P\in \mathbb{A}^n$
we define $\sigma(P) = (\sigma(x_1),\ldots,\sigma(x_n))$. Now we define
the set of $k$-rational points in $\mathbb{A}^n$ to be those fixed under action by
the galois group
$$ \mathbb{A}^n(k) = \{ P \in \mathbb{A}^n : \sigma(P) = P\, \forall\, \sigma \in
Gal(\bar{k}/k) \} $$

Similarly we define the set of $k$-rational points in $\mathbb{P}^n$ to be
$$ \mathbb{P}^n(k) = \{ P \in \mathbb{P}^n : \sigma(P) = P\, \forall\, \sigma \in 
Gal(\bar{k}/k) \} $$

\begin{mydef}
 A polynomial $f\in\bar{k}[X]$ is said to be \emph{homogeneous of degree $d$} if for all
$\lambda\in\bar{k}$ we have.
$$f(\lambda x_0,\ldots,\lambda x_n) = \lambda^d f(x_0,\ldots,x_n)$$
Furthermore an ideal $I\subseteq\bar{k}[X]$ is said to be homogeneous if it is generated
by homogeneous polynomials.
\end{mydef}

\begin{mydef}
 A \emph{projective algebraic set} is of the form
$$ V_I = \{ P\in \mathbb{P}^n : f(P) = 0\, \forall homogeneous\, f\in I \} $$
Given such a set $V$ we associate to it an ideal $I(V) \in \bar{k}[X]$ generated by
$$ \{f\in\bar{k} : f\, homogeneous\, and\, f(P)=0\, \forall P\in V \} $$
\end{mydef}

\begin{mydef}
 A projective algebraic set is called a \emph{projective variety} if the homogeneous
ideal defined above is a prime ideal in $\bar{k}[x]$.
\end{mydef}

\begin{mydef}
 Let $V/k$ be a projective variety (i.e. V defined over $k$), then the projective coordinate
ring of $V/k$ is defined by
$$ k[V] = \frac{k[x]}{I(V/k)}$$
Note that since $I(V/k)$ is a prime ideal, the coordinate ring is an integral domain.
This enables us to form the quotient field of $k[V]$ which we denote $k(V)$, and it is called
the \emph{function field} of $V$.
\end{mydef}

A rather interesting ideal to keep in mind is given by
$$ M_p = \{ f\in \bar{k}[V] : f(P)=0 \} $$
This is a maximal ideal because the map $\phi: \bar{k}[V] \rightarrow \bar{k}$ given by
$ f \mapsto f(P) $ has kernel exactly $M_p$. It is clearly onto, so it induces an
isomorphism $$\tilde{\phi}: \bar{k}[V]/M_p \rightarrow \bar{k} $$

\begin{mydef}
 The \emph{localization of $\bar{k}[V]$ at $M_p$} is given by
$$ \bar{k}[V]_P = \{ h \in \bar{k}[V] : h = f/g\, f,g\in \bar{k}[V]\, and\, g(P)=0 \} $$
The functions in $\bar{k}[V]_P$ are all defined at $P$.
\end{mydef}

\begin{ex}
 If $V$ is a variety given by a single non-constant polynomial equation
$$f(x_1,\ldots,x_n) = 0$$ 
then the dimension of the variety $dim(V)$ is $n-1$. The (projective) varieties
we will study are called \emph{elliptic curvevs} and are
given by polynomial equations
$$E: y^2 = x^3+ax+b$$
They correspond to polynomials of the form $f(x,y) = x^3+ax+b-y^2$ so $dim(E)=1$.
We say curves are projective varieties of dimension $1$.
\end{ex}

The objects we will be working on are projective varieties, but they are not
very interesting unless we define maps between them.

\begin{mydef}
 Let $V_1$ and $V_2$ be projective varieties, a \emph{rational map} $\phi: V_1 \rightarrow V_2$
is a set of maps $\{\phi_0,\ldots,\phi_n\}$ with $\phi_i \in \bar{k}(V_1)$ such that for every
$P\in V_1$ we define
$$\phi(P) = [\phi_0(P),\ldots,\phi_n(P)] \in V_2$$
Such a rational map is called a \emph{morphism} if it is defined at every point $P$.
\end{mydef}

The varieties and the morphisms between them make up a category, so our next
definition of an isomorphism will be the general one found in category theory.

\begin{mydef}
 Two varieties $V$ and $W$ are \emph{isomorphic} denoted $V\simeq W$
if there exist morphisms $\phi: V \rightarrow W$ and $\psi: W \rightarrow V$ such that
$\phi \psi = 1_W$ and $\psi \phi = 1_V$.
If the rational functions $\psi$ and $\phi$ are defined over $k$ we say that $V$ and $W$
are isomorphic over $k$. If not, they are isomorphic over some field extension of $k$
(i.e. $\bar{k}$).
\end{mydef}


\subsection{Curves and divisors}
Recall that curves are projective varieties of dimension one. Even more special
are elliptic curves, which are curves with \emph{genus} equal to 1. This will
be introduced later on. These are in practise the only curves we will be working with.

\begin{mydef}
 Let $C$ be a curve and $P\in C$ a non-singular point on the curve. A valuation on
$\bar{k}[C]_P$ is given by
$$ ord_P : \bar{k}[C]_P \rightarrow \{ 0, 1, 2, \ldots \} \cup \{ \infty \} $$
$$ ord_P(f) = max \{ d\in \mathbb{Z} : f\in M_P^d \} $$
This is called \emph{the order of $f$ at $P$}.
Letting $ord_P(f/g) = ord_P(f) - ord_P(g)$ we can extend the definition to the entire
quotient ring $\bar{k}(C)$
$$ ord_P: \bar{k}(C) \rightarrow \mathbb{Z}\cup \{\infty \} $$
\end{mydef}

The definition of order agrees with the one found in complex analysis.
If $ord_P(f) < 0$ f has a pole at $P$ and we write $f(P)=\infty$. 
If $ord_P(f) \ge 0$ f has a zero and is defined at $P$, so $f(P)$ can be calculated. 

\begin{prop}
 Let $C$ be a smooth curve. If $f\in \bar{k}(C)$ is not the constant function, then
$f$ has finitely many poles and zeros.
\label{prop:1}
\end{prop}
\begin{proof}
 FIXME. Prop 1.2 AEC.
\end{proof}

\begin{mydef}
 The \emph{divisor group} of a curve $C$ is the free abelian group generated by
points of $C$, denoted $Div(C)$. A divisor $D\in Div(C)$ is of the form
$$ D = \sum_{P\in C} n_P(P)$$
with $n_P\in\mathbb{Z}$ and $n_P = 0$ for almost all $P$.
\end{mydef}

With this in mind we can define the degree of a divisor as the sum of its 
coefficients. We also define the sum of a divisor as the sum in the group $E(\bar{k})$, so
$$ deg(D) = deg(\sum_{P\in C} n_P(P)) = \sum_{P\in C} n_P \in \mathbb{Z}$$
$$ sum(D) = sum(\sum_{P\in C} n_P(P)) = \sum_{P\in C} n_P P \in E(\bar{k})$$

These functions enable us to define the subgroup of divisors of degree zero,
$Div^0(C) \subset Div(C)$, so $Div^0(C) = \{ D\in Div(C) : deg(D) = 0 \}$.

Now let $C$ be a smooth curve and $f\in \bar{k}(C)$ a non-zero function. Since $f$
has finitely many poles and zeros (Prop. \ref{prop:1}) we can define the divisor of a
function as
$$ div(f) = \sum_{P\in C} ord_P(f)(P) $$
Note that $ord_P$ is a valution we have $ord_P(fg) = ord_P(f)+ord_P(g)$
for non-zero $f,g\in \bar{k}(C)$. Thus we get a group homomorphism
$$ div: \bar{k}(C)^* \rightarrow Div(C)$$

\begin{mydef}
 The \emph{principal divisors} of $C$ are the divisors of the form
$ D = div(f) $ for some non-zero $f\in \bar{k}(C)$. This is exactly
the image of the function $div$ and we denote this set by $Prin(C)$.
Note that since divisors of rational functions have the same number of poles
and zeros (when counted correctly), we have $deg(div(f)) = 0$. [EGET TEOREM?]
\end{mydef}

Two divisors are said to be \emph{equivalent} denoted $D_1 \sim D_2$ if
their difference is a principal divisor, $D_1 - D_2 = div(f)$ for some $f$. In
addition we can put a partial ordering on $Div(C)$, saying that a divisor $D$ is
\emph{positive} $\sum n_P(P)=D \geq 0$ if $n_P \geq 0$ for every $P\in C$. Furthermore
we write $D_1 \geq D_2$ to indicate that $D_1 - D_2$ is positive.

\begin{ex}
 Inequalities can easily summarize some key properties of a function. So instead of
saying $f \in \bar{k}(C)$ is regular everywhere except at $P$ and $Q$, where it has a
pole and a root of order $m$ and $n$ respectively, we could write
$$ div(f) \geq -m(P)+n(Q) $$
\end{ex}

The last example motivates our next definition, where we collect all functions
which satisy some inequality. This turns out to make up a finite dimensional
$\bar{k}$-vector space.

\begin{mydef}
 Let $D \in Div(C)$ be a divisor, and we define the set of functions
$$ \mathscr{L}(D) = \{ f\in \bar{k}(C) : div(f) \geq -D \} \cup \{ 0\} $$
\end{mydef}

\begin{prop}
 $\mathscr{L}(D)$ is a finite dimensional $\bar{k}$-vector space, and we denote
its dimension by
$$ \ell(D) = dim_{\bar{k}} \mathscr{L}(D) $$
\end{prop}
\begin{proof}
 First note that if $D' > D$ then $D' = D + P_1 \ldots P_s$, so we get an ascending chain of
subspaces
$$ \mathscr{L}(D) \subseteq \mathscr{L}(D+P_1) \subseteq \ldots \subseteq \mathscr{L}(D+P_1 \ldots P_s) $$

\end{proof}

\begin{mydef}
 The space of differential forms on a curve $C$ is a $\bar{k}(C)$-vector space denoted $\Omega_C$
generated by symbols subject to the releations known from analysis. 
For $x, y \in \bar{k}(C)$ and $a \in \bar{k}$
\begin{enumerate}
  \item d(x+y) = dx + dy
  \item d(xy) = xdy + ydx
  \item da = 0
\end{enumerate}
Let $f_i \in \bar{k}(C)$ and $dx_i$ be the symbols as defined above, a general
element $\omega \in \Omega_C$ is of the form
$$ \omega = \sum f_i dx_i $$
\end{mydef}

Divisors in the image of the map $div: \Omega_C \rightarrow Pic(C)$ are called
\emph{canonical divisors}. They will play a role in the next theorem which will
serve as an important tool for calculating the dimension of the vector space
$\mathscr{L}(D)$, which will be crucial in establishing an important isomorphism.
It will also serve as a definition of the genus $g$.

\begin{thm}
 \textbf{(Riemann-Roch)}
  Let $C$ be a smooth curve and $K_C$ a canonical divisor on $C$. Then for
any $D \in Div(C)$ we have
$$ \ell(D) - \ell(K_C - D) = deg(D) - g + 1 $$
where $g \geq 0$ is called the \emph{genus} of the curve $C$.
\end{thm}
\begin{proof}
 A proof would be outside the scope of this paper + referanser.
\end{proof}

Given a non-constant map of curves $\phi: C_1 \rightarrow C_2$, we have an induced map
on function fields $\phi^*: K(C_2) \rightarrow K(C_1)$. From this again we get an induced
map on differential forms
$$ \phi^*: \Omega_{C_2} \rightarrow \Omega_{C_1} $$
$$ \phi^*\left(\sum f_i dx_i\right) = \sum (\phi^* f_i) d(\phi^* x_i) $$

\begin{mydef}
 Let $\phi: C_1 \rightarrow C_2$ be a map of curves and $\phi^*$ its induced map on
function fields. We then say that $\phi$ is a \emph{seperable map} if
$K(C_1)/\phi^* K(C_2)$ is a seperable extension.
\end{mydef}

Recall that a field extension is separable if and only if the derivative of the minimal
polynomial for each element is non-zero. This fact is the motivation for our next result,
which gives a useful criterion for determining when a map is separable.

\begin{prop}
 Let $\phi: C_1 \rightarrow C_2$ be a map of curves, then $\phi$ is separable if and only if
the induced map $\phi^*: \Omega_{C_2} \rightarrow \Omega_{C_1} $ is non-zero.
\end{prop}

\begin{thm}
 $$ sum: Pic^0(C) \rightarrow E(\bar{k}) $$
is a group isomorphism
\end{thm}
\begin{proof}
 We begin by showing that there is a unique point $P \in E(\bar{k})$ associated to
each $D \in Div^0(E)$ as follows
$$ D \sim (P) -(O) $$
This will be given by a map
$$ \sigma: Div^0(E) \rightarrow E(\bar{k}) $$
From [referanse] we have that $\ell(D+(O)) = deg(D+(O)) = 1$ since $deg(D) = 0$.
Let then $f \in \bar{k}(E)$ be a generator for $\mathscr{L}(D+(O))$, so by definition
$$ div(f) \geq -D-(O) $$
But since $deg(div(f)) = 0$ and $deg(-D-(O)) = -1$ we have for some $P \in E(\bar{k})$ that
$$ div(f) = -D-(O)+(P) $$
which is exactly the definition of
$$ D \sim (P) - (O) $$

This point $P$ is unique, because if we assume that $P'$ is another point with the same
property, then
$$ (P) \sim D + (O) \sim (P') $$
so by [3.3 korrolar til riemann-roch]
$ P = P'$.

The map $\sigma$ is easily seen to be a surjection, because for any $P \in E(\bar{k})$ we have
$$ \sigma((P)-(O)) = P $$

Now if we can show that the kernel of $\sigma$ is exactly the principal divisors we are done.
Let us assume that $\sigma(D) = O$ so from definition we have that $D \sim (O)-(O) \sim (O)$
meaning $D - (O) = div(f)$ for some $f \in \bar{k}(E)$, so $D = div(f)$ is principal.
For the other implication we assume that $D = div(f)$ is principal. Using the definition and
letting $P$ be any point and $f, f' \in \bar{k}(E)$ a calculation yields
$$ \sigma(D) = \sigma(div(f)) = (P)-(O) $$
$$ div(f) \sim (P) - (O) $$
$$ div(f) - (P) - (O) = div(f') $$
$$ (O) - (P) = div(f') - div(f) = div(f' f) $$
So $ (P) \sim (O) $ which implies $P = O$ from [3.3-korrolaret].

We have thus established the group isomorphism which we again will call $\sigma$
$$ \sigma : Div^0(E)/Prin(E) = Pic^0(E) \rightarrow E(\bar{k}) $$

\end{proof}
