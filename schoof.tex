\section{Schoof's algorithm and improvements}
The idea of Schoof's algorithm is to calculate the Frobenius trace modulo small primes,
then assemble this information using the Chinese remainder theorem. Choosing the set
of small primes such that their product $N > 4 \sqrt q$ (with $q$ the size of our
field) gives us the trace $t$ modulo $N$, which by the Hasse bound is exactly
the Frobenius trace.

Recall for this section that an elliptic curve corresponds to a lattice $\Lambda$
so we have an isomorphism
$$ \bar{k}/\Lambda \simeq E(\bar{k}) $$
$$ z \mapsto (\wp(z), \wp '(z)) $$
where $\wp(z)$ is the elliptic Weierstrass function
$$\wp(z) = \frac{1}{z^2} + \sum_{k=1}^\infty c_k z^{2k}. $$

\begin{mydef}
 The \emph{division polynomials} are polynomials $\Psi_n(x,y) \in \mathbb{Z}[x,y,A,B]$
defined by the recurrence relations
\begin{align*}
  \Psi_0 &= 0 \\
  \Psi_1 &= 1 \\
  \Psi_2 &= 2y \\
  \Psi_3 &= 3x^4 + 6Ax^2 + 12Bx - A^2 \\
  \Psi_{2n+1} &= \Psi_{n+2} \Psi_n^3 - \Psi_{n+1}^3 \Psi_{n-1} \\
  \Psi_{2n}   &= (2y)^{-1} \Psi_n(\Psi_{n+2} \Psi_{n-1}^2 - \Psi_{n-2} \Psi_{n+1}^2)
\end{align*}
where $\Psi_n(x,y) = 0$ is and only if $(x,y) \in E[n]$.
\end{mydef}
The construction of these polynomials can be done in at least two ways and I will discuss
both of them briefly.

One way of doing this is to construct
a function having poles at the $n$-torsion points of our elliptic curve as follows
$$ f_n(z) = n^2 \prod(\wp(z) - \wp(u)) $$
where the product is taken over all $n$-torsion points of $\bar{k}/\Lambda$, denoted
$\bar{k}/\Lambda[n]$. This function has roots at exactly the $n$-torsion points by definition,
which is at least what we want. A more thorough examination of this method can be found
in \cite{Lang}.
Another way which is more elementary but highly computational is to work explicitly
with the addition formulas for elliptic curves.

Replacing the terms $y^2$ in $\Psi_n$ by $x^3 + Ax + B$ we obtain polynomials $\Psi_n '$ in
$\mathbb{F}_q[x]$ if is $n$ is odd or $y \mathbb{F}_q[x]$ if $n$ is even. To avoid
this distinction we define
$$
f_n(x) = \begin{cases}
          \Psi_n '(x,y) & \text{if n is odd.} \\
	  \Psi_n '(x,y)/y & \text{if n is even.}
         \end{cases}
$$


\begin{prop}
 Let $n \geq 2$ and $\Psi_n$ the division polynomial as defined above, then
$$ nP = (x - \frac{\Psi_{n-1} \Psi_{n+1}}{\Psi_n^2}, \frac{\Psi_{n+2} \Psi_{n-1}^2 - \Psi_{n-2} \Psi_{n+1}^2}{4y \Psi_n^3} ).$$
\end{prop}

\subsection{Schoof's algorithm}
For an elliptic curve over $\mathbb{F}_q$ given by
$$ E: y^2 = x^3 + Ax + B $$
we want to compute the size of $\#E(\mathbb{F}_q)$, we know from before that
$$ \#E(\mathbb{F}_q) = q + 1 - t $$
where $t$ is the trace of the Frobenius as seen in chapter \ref{frob}. We know
that $t$ satisfies the Hasse bound namely
$$ \lvert \#E(\mathbb{F}_q)-q-1\rvert=\lvert t\rvert < 2\sqrt{q}.$$
Let $S = \left\{3, 5, 7, 11, \ldots \, L \right\}$ be the set of odd primes $\leq L$ such
that the product is bigger than the Hasse interval
$$ N = \prod_{\ell \in S} \ell  > 4\sqrt{q}. $$
If we can then calculate $t\, (mod\,\ell)$ for all $\ell \in S$ we can uniquely
determine $t\,(mod\,N)$ by invoking the Chinese remainder theorem,
which then by the Hasse bound is our Frobenius trace $t$.

We will now look at
how to calculate $t\, (mod\,\ell)$. Let $\phi$ be the Frobenius endomorphism
restricted to $E[\ell]$ and let $q_\ell$, $\tau$ be $q$ and $t$ reduced modulo $\ell$
respectively. The computation of $\tau$ can then be done by checking if
$$ \phi^2(P) + q_\ell P = \tau \phi(P) $$
holds for $P \in E[\ell]$. To perform the addition on the left hand side of the
equality we need to distinguish the cases where the points are on a vertical line or not.
In other words we have to verify if for $P = (x,y) \in E[\ell]$ the following holds
$$ \phi^2 (P) = \pm q_\ell P. $$
Noting that $-P = (x, -y)$ we write out the equality for the $x$-coordinates in terms of
division polynomials
$$ x^{q^2} = x - \frac{\Psi_{q_\ell-1} \Psi_{q_\ell+1}}{\Psi_{q_\ell}^2}(x,y). $$
Writing this out in terms of $f_n(x)$ and noting that for $n$ even we have
$\Psi_n(x,y) = y f_n(x)$, a calculation for $q_\ell$ even yields
\begin{eqnarray*}
 x^{q^2} &=& \frac{f_{q_\ell-1}(x) f_{q_\ell+1}(x)}{(f_{q_\ell}(x) y)^2} \nonumber \\
	 &=& \frac{f_{q_\ell-1}(x) f_{q_\ell+1}(x)}{f_{q_\ell}^2(x) (x^3+Ax+B)} \nonumber \\
\end{eqnarray*}
The calculation for $q_\ell$ odd is similar and we get the equality

$$
x^{q^2} = \begin{cases}
           x - \frac{f_{q_\ell-1}(x) f_{q_\ell+1}(x)}{f_{q_\ell}^2(x) (x^3+Ax+B)} & \text{if } q_\ell \text{ is even} \\
	   x - \frac{f_{q_\ell-1}(x) f_{q_\ell+1}(x) (x^3+Ax+B)}{f_{q_\ell}^2(x)} & \text{if } q_\ell \text{ is odd} 
          \end{cases}
$$
We thus get two equations and we want to verify they have any solutions $P \in E[\ell]$. For
doing this we compute the following greatest common divisors
$$ \gcd((x^{q^2} - x)f_{q_\ell}^2(x) (x^3+Ax+B)+f_{q_\ell-1}(x) f_{q_\ell+1}(x), f_\ell(x)) \quad (q_\ell \text{ even)}$$
$$ \gcd((x^{q^2} - x)f_{q_\ell}^2(x)+f_{q_\ell-1}(x) f_{q_\ell+1}(x) (x^3+Ax+B), f_\ell(x)) \quad (q_\ell \text{ odd)}$$
We are now going to treat the rest in two cases, depending on the value from the above gcds.

\textbf{Case 1:} $\gcd \neq 1$ meaning there exist a non-zero $\ell$-torsion point $P$ such that $\phi^2(P) = \pm q_\ell P$.
If $\phi^2 (P) = -q_\ell P$ we have that $\tau \phi(P) = 0$ but since $\phi(P) \neq 0$ we know that $\tau = 0$.
If $\phi^2(P) = q_\ell P$ we have that 
$$ 2 q_\ell P = \tau \phi(P) \Leftrightarrow \phi(P) = \frac{2 q_\ell}{\tau}. $$
Substituting the last equality into $\phi^2(P) = q_\ell P$ we obtain
$$ \frac{4 q_\ell^2}{\tau^2} = q_\ell P \Leftrightarrow 4 q_\ell P = \tau^2 P. $$
We thus obtain the congruence $\tau^2 \equiv 4q_\ell \quad (mod\, \ell)$

\textbf{Case 2:} $\gcd = 1$ so $\phi^2(P) \neq \pm q_\ell P$ meaning the two points are 
not equal nor are they on the same vertical line for any $\ell$-torsion point $P$. 
This enables us to do the addition $\phi^2(P) + q_\ell P$ using the appropriate addition formulas.
Recall that if $P = (x_1, y_1)$ and $Q = (x_2, y_2)$ are two points on $E$ with
$P \neq Q$ we have that their sum is given by $P+Q = (x_3, y_3)$ where
$$ \lambda = \frac{y_2 - y_1}{x_2 - x_1} $$
$$ x_3 = -x_1 - x_2 + \lambda^2 $$
$$ y_3 = -y_1 -\lambda(x_3 - x_1).$$
We can now write out the addition explicitly in terms of polynomials as follows
$$ \lambda = \frac{\Psi_{q_\ell+2} \Psi_{q_\ell-1}^2 - \Psi_{q_\ell-2}\Psi_{q_\ell+1}^2 - 4y^{q^2+1}\Psi_{q_\ell}^3}
		  {4\Psi_{q_\ell} y ((x-x^{q^2})\Psi_{q_\ell}^2 - \Psi_{q_\ell-1}\Psi_{q_\ell+1}} $$
with the left hand side given by

$$\phi^2 (P) + q_\ell P = \left(-x^{q^2}-x+\frac{\Psi_{q_\ell-1}\Psi_{q_\ell+1}}{\Psi_{q_\ell}^2}+\lambda^2,
		     -y^{q^2}-\lambda\left(-2x^{q^2}-x+\frac{\Psi_{q_\ell-1}\Psi_{q_\ell+1}}{\Psi_{q_\ell}^2}\right)\right).$$
The right hand side is as before given by
$$ \tau\phi(P)=\left(x^q-\left(\frac{\Psi_{\tau+1}\Psi_{\tau-1}}{\Psi_\tau^2}\right)^q,\left(\frac{\Psi_{\tau+2}\Psi_{\tau-1}^2 - \Psi_{\tau-2}\Psi_{\tau+1}^2}{4y\Psi_\tau^3}\right)^q\right) $$
So figuring out if $$\phi^2(P) + q_\ell P = \tau \phi(P) $$ amounts to checking if the difference
of the two expressions are zero modulo the division polynomials for some $0 \leq \tau < \ell$.
More explicit formulas can be found in \cite{Schoof85}.

\subsection{Schoof-Elkies algorithm}
When doing the calculations in Schoof's algorithm we were working modulo the division
polynomials $\Psi(x,y)$ of degree $\ell^2-1$. Instead we can exploit some special primes
called \emph{Elkies primes} that enables us to work in a cyclic subgroup $C$ of $E[\ell]$.
Here $C$ will correspond to a $1$-dimensional eigenspace.

The Frobenius endomorphism restricted to $E[\ell]$ satisfies the characteristic equation
$$ \phi^2 - \tau \phi + q_\ell = 0 $$
where $\tau$ and $q_\ell$ is as before. The roots of this equations are the eigenvalues
of $\phi$ restricted to $E[\ell]$ and they are given by
$$ \lambda_{1,2} = \frac{\tau \pm \sqrt{\tau^2 - 4 q_\ell}}{2}. $$
If the discriminant $\tau^2 - 4 q_\ell$ is a square modulo $\ell$ we have that
$\lambda_{1,2} \in \mathbb{F}_q$.
\begin{mydef}
A prime $\ell$ such that $\tau^2 - 4 q_\ell$ is a square modulo $\ell$ is called
an \emph{Elkies prime}.
\end{mydef}
For primes of this type we obtain a factorization
$$ (\phi - \lambda_1)(\phi - \lambda_2) = 0 $$
so for an eigenvalue $\lambda$ we have that $\phi(P) = \lambda P$ for a point $P$. Thus
$P$ is the generator for a cyclic eigenspace $C \subset E[\ell]$ of order $\ell$
corresponding to $\lambda$.
Notice that we have an exact sequence of groups
$$ 0 \rightarrow C \rightarrow E \rightarrow E/C \rightarrow 0 $$
where the map $E \rightarrow E/C$ has cyclic kernel $C$ of order $\ell$.
Determining which primes are Elkies primes can be done by working with the
modular polynomials. From Theorem \ref{modpol} we have that $\Phi_\ell(j(E),j(E/C)) = 0$,
so letting the isogeny $f: E \rightarrow E'$ have cyclic kernel $C$ we get an exact
sequence
$$ 0 \rightarrow C \rightarrow E \rightarrow E' \rightarrow 0 $$
which by a diagram chase yields $ E' \simeq E/C $. This argument gives us the following result
\begin{prop}
 $ \Phi_\ell(j(E), x) = 0$ for $x \in \mathbb{F}_q$ if and only if $\ell$ is an Elkies
prime.
\end{prop}
Figuring out if $\ell$ is an Elkies prime can thus be done fast by calculating
$$\gcd(\Phi_\ell(j(E), x), x^q - x).$$
Now since we are working only with primes of this type we restrict ourself to working
in the subspace $C$ of order $\ell$. There is thus a factor $G_\ell(x)$ of the division polynomial
which has the $x$-coordinates of points in $C$ as roots. Since
similar points in $C$ of different sign are on the same vertical line we only include
unique points up to sign. In this way we get that the degree of $G_\ell(x)$ is $\frac{\ell-1}{2}$.

From the theory of eigenvalues we know that if $\lambda_1, \lambda_2$ are
eigenvalues of $\phi$ then $$tr(\phi) = \lambda_1 + \lambda_2$$
We also know using Proposition \ref{detdeg} that $$\lambda_1 \lambda_2 = \det \phi = q $$
Using this we can recover the trace of $\phi$ by calculating one of the eigenvalues
$$ \tau \equiv \lambda + \frac{q}{\lambda} \quad (mod\, \ell). $$
To compute the eigenvalue $\lambda$ we can thus check which of the relations
$$ \phi(P) = (x^q, y^q) = \lambda P$$
holds on the eigenspace $C$, this mean we can work modulo $G_\ell(x)$. This enables us
to work in the much smaller ring $$\mathbb{F}_q[x,y]/(G(x), y^2 - x^3 - Ax - B) $$
and thus greatly improves Schoof's original approach.

The obstacle that remains is how we can possible calculate the factor 
$$ G_\ell(x) = \prod_{(x',y')\in C} (x-x')$$
of the division polynomial where the product is taken over all 
unique points $P = (x', y')$ up to sign. When calculating the gcd
$\gcd(\Phi_\ell(j(E), x), x^q - x)$ we obtain a polynomial whose roots
(at most two) are the $j$-invariants of the $\ell$-isogenous curves
$\tilde{E} = E/C$ where $C$ is the eigenspace corresponding to $\lambda$.
The next theorem enables us to calculate an explicit formula for the Weierstrass
equation of $\tilde{E}$.

\begin{thm}
 Let $E$ be given by the equation
$$ E: y^2 = x^3 + Ax + B $$
with $j = j(E)$. Then the equation for the $\ell$-isogenous curve $\tilde{E}$ with
$\tilde{j} = j(\tilde{E})$ is given by
$$ \tilde{E}: y^2 = x^3 + \bar{A}x + \bar{B} $$

$$\bar{A} = -\frac{\tilde{j}'^2}{48 \tilde{j}(\tilde{j} - 1728)} \quad
  \bar{B} = -\frac{\tilde{j}'^3}{864 \tilde{j}^2(\tilde{j} - 1728)}. $$

And letting $$\Phi_{\ell, x} = \frac{\partial \Phi_\ell}{\partial x} \quad
              \Phi_{\ell, y} = \frac{\partial \Phi_\ell}{\partial y}$$
be the partial derivatives with respect to $x$ and $y$ respectively we have that
$$ \bar{j}' = -\frac{18 B \Phi_{\ell, x}(j, \bar{j})}{\ell A \Phi_{\ell, y}(j, \bar{j})} j. $$
\end{thm}
The next theorem will enable us to compute the sum of the $x$-coordinates of the points in
our subspace $C$. This value will be used to calculate every coefficient of $G_\ell(x)$, notice
that if we formally multiply out the product of $G_\ell(x)$ we get
$$ G_\ell(x) = x^\frac{\ell-1}{2} - \frac{p_1}{2} x^\frac{\ell-3}{2} + \ldots. $$
Here $p_1$ is the sum of the $x$-coordinates of $C$, the division by two is because they
appear twice as a result of the symmetry around the $x$-axis.
\begin{thm}
 Given our two curves $E: y^2 = x^3 + Ax + B$ and $\tilde{E}: y^2 = x^3 + \bar{A}x + \bar{B}$ we let
$E_4 = -48A$, $E_6 = 864B$ and similarly for our $\ell$-isogenous curve $\bar{E_4} = -48\bar{A}$,
$\bar{E_6} = 864\bar{B}$. Then we obtain an explicit formula for
$$ p_1 = \sum_{(x,y)\in C} x $$
$$ p_1 = \frac{\ell}{2}J + \frac{\ell}{4} \left( \frac{E_4^2}{E_6} - \ell \frac{\bar{E_4^2}}{\bar{E_6}} \right) $$
where by using the usual partial derivative notation $\Psi_{\ell, xx} = \frac{\partial^2 \Psi_\ell}{\partial x^2}$
etc. we write $J$ as 
$$ J = -\frac{j'^2 \Psi_{\ell, xx}(j, \tilde{j}) + 2\ell j' \tilde{j'} \Psi_{\ell, xy}(j, \tilde{j})
+\ell^2 \tilde{j'}^2 \Psi_{\ell, yy}(j, j')}{j' \Psi_{\ell, x}(j, j')}. $$
Here $j'= -j \frac{E_6}{E_4}$ and similarly $\tilde{j'} = -\tilde{j} \frac{\tilde{E_6}}{\tilde{E_4}}$.
\end{thm}

Given the value of $p_1$ we can calculate the rest of the coefficients using a theorem from \cite{Schoof}.
\begin{thm}
 Let $\ell \neq char(k)$ be a prime, and $\phi: E \rightarrow \widetilde{E}$ be an isogeny with $\ker(\phi)$
cyclic of size $\ell$, then the polynomial $G_\ell(x)$ which vanishes on the $x$-coordinates of
elements of $\ker(\phi)$ satisfies
$$ z^{\ell-1}G_\ell(\wp(z)) = exp(-\frac{1}{2}p_1 z^2 - \sum_{k=1}^\infty \frac{\widetilde{c}_k-\ell c_k}{(2k+1)(2k+2)}z^{2k+2}).$$
\end{thm}
\begin{proof}
 The proof of Schoof uses analytic theory heavily, he introduces the Weierstrass $\zeta$-function
which is defined by
$$\zeta(z) = \frac{1}{z} - \sum_{k=1}^\infty \frac{c_k}{2k+1} z^{2k+1}. $$
Differentiating we see that $\zeta'(z) = - \wp(z)$. Let $\Lambda = \omega_1 \mathbb{Z}+\omega_2\mathbb{Z}$
be the lattice corresponding to $E$ and then we have that
$\widetilde{\Lambda} = \frac{\omega_1}{\ell}\mathbb{Z}+\omega_2\mathbb{Z}$ is the lattice corresponding
to $\widetilde{E}$. Setting $\zeta$ and $\widetilde{\zeta}$ to be the Weierstrass $\zeta$-functions for
$E$ and $\widetilde{E}$ respectively, Schoof eventually arrives at the equality
$$-\ell \zeta(z)+\widetilde{\zeta}(z)-p_1 z = \sum_{i=1}^{(\ell-1)/2} \frac{\wp'(z)}{\wp(z)-\wp(\frac{i}{\ell}\omega_1)}. $$
Notice that $\frac{d}{dz} \wp(z) - \wp(\frac{i}{\ell}\omega_1) = \wp'(z)$ so we can invert the process
of logarithmic differentiation
$$ \frac{df}{dz} = \frac{f'}{f}$$
on both sides of the equality, setting $$f(z) = \prod_{i=1}^{(\ell-1)/2}(\wp(z)-\wp(\frac{i}{\ell}\omega_1).$$

A thorough proof can be found in \cite{Schoof}.
\end{proof}

The coefficients of $G_\ell(x)$ can thus be obtained by expanding both sides of
the equality from the previous theorem and comparing the coefficients of like powers of $z$.
Setting $w=z^2$ and letting $A(w)$ be the function on the right-hand side of the equality
expanded as a power series in $w$. Also let $C(w) = \wp(z) - \frac{1}{w} = \sum_{k=1}^\infty c_k w^k$,
the Weierstrass $\wp$-function with the first term removed. For notational convenience we
write $[B(w)]_j$ for the coefficient of $w^j$ in the power series $B(w)$. Letting $g_i$ be the
coefficient of $x^i$ in $G_\ell(x) = x^d + \sum_{i=0}^{d-1} g_i x^i$ with $d = \frac{\ell-1}{2}$ 
we get the recursion
$$g_{d-i} = [A(w)]_i - \sum_{k=1}^i \left( \sum_{j=0}^k \binom{d-i+k}{k-j} [C(w)^{k-j}]_j \right) g_{d-i+k} $$
as seen in \cite{Blake}.

We end this section by given a short summary of the Schoof-Elkies algorithm:
choose $S$ to be the set of small odd primes such that the product is bigger than the Hasse
interval. For each such prime $\ell \in S$ we check if the following holds
\begin{equation} \label{eq1}
 \phi^2(P) + q_\ell P = \tau \phi(P)
\end{equation}

for all $P\in E[\ell]$. Here the addition is done using the appropriate addition law on $E$,
treated in special cases depending on whether the two points are on the same vertical line
or not. The points $q_\ell P$ and $\tau \phi(P)$ can be calculated by using properties of
the division polynomials. We then proceed by checking if \eqref{eq1} holds for $P\in E[\ell]$ and
$ 0 \leq \tau < \ell$, all calculations done modulo the division polynomials $\Psi_\ell(x)$
and the curve equation for $E$. This gives an algorithm of complexity $O(\log^8 q)$ \cite{Schoof}.

The improvement due to Elkies is the restriction of the set of small primes $S$. Choosing
only the small primes $\ell$ such that $\tau^2 - 4q_\ell$ is a square modulo $\ell$ (Elkies primes),
we obtain a factorization of the characteristic equation of the Frobenius. This thus
gives you a cyclic eigenspace $C\subset E[\ell]$ of dimension $1$ corresponding to an eigenvalue
$\lambda$. Computing one of the eigenvalues $\lambda$ will give us the trace modulo $\ell$.
We thus have to check if
$$ \phi(P) = \lambda P$$
holds on $C$. This enables us to use a factor of the division polynomial $G_\ell(x)$ of degree
$\frac{\ell-1}{2}$. The improvement is thus to do the calculations of $x^q, y^q$ etc.
modulo $G_\ell(x)$ of much smaller degree, instead of working with the full division polynomial $\Psi_\ell(x)$.
Schoof-Elkies can thus be shown \cite{Schoof} to have complexity $O(\log^5 q)$.