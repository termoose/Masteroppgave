\section{Schoof's algorithm and improvements}
\subsection{Division polynomials}
Recall for this section that an elliptic curve corresponds to a lattice $\Lambda$
so we have an isomorphism
$$ \bar{k}/\Lambda \simeq E(\bar{k}) $$
$$ z \mapsto (\wp(z), \wp '(z)) $$
where $\wp(z)$ is the elliptic Weierstrass function. 

\begin{mydef}
 The \emph{division polynomials} are polynomials $\Psi_n(x,y) \in \mathbb{Z}[x,y,A,B]$
defined by the recurrence relations
\begin{align*}
  \Psi_0 &= 0 \\
  \Psi_1 &= 1 \\
  \Psi_2 &= 2y \\
  \Psi_3 &= 3x^4 + 6Ax^2 + 12Bx - A^2 \\
  \Psi_{2n+1} &= \Psi_{n+2} \Psi_n^3 - \Psi_{n+1}^3 \Psi_{n-1} \\
  \Psi_{2n}   &= (2y)^{-1} \Psi_n(\Psi_{n+2} \Psi_{n-1}^2 - \Psi_{n-2} \Psi_{n+1}^2)
\end{align*}
where $\Psi_n(x,y) = 0$ is and only if $(x,y) \in E[n]$.
\end{mydef}
The construction of these polynomials can be done in at least two ways and I will discuss
both of them briefly.

One way of doing this is to construct
a function having poles at the $n$-torsion points of our elliptic curve as follows
$$ f_n(z) = n^2 \prod(\wp(z) - \wp(u)) $$
where the product is taken over all $n$-torsion points of $\bar{k}/\Lambda$, denoted
$\bar{k}/\Lambda[n]$. This function has roots at exactly the $n$-torsion points by definition,
which is at least what we want. A more throrough examination of this method can be found
in [serge lang-ref].

Another way which is more elementary by highly computational is to work explicitly
with the addition formulas for elliptic curves. + mer forklaring.

Replacing the terms $y^2$ in $\Psi_n$ by $x^3 + Ax + B$ we obtain polynomials $\Psi_n '$ in
$\mathbb{F}_q[x]$ if is $n$ is odd or $y \mathbb{F}_q[x]$ if $n$ is even. To avoid
this distinction we define
$$
f_n(x) = \begin{cases}
          \Psi_n '(x,y) & \text{if n is odd} \\
	  \Psi_n '(x,y)/y & \text{if n is even}
         \end{cases}
$$


\begin{prop}
 Let $n \geq 2$ and $\Psi_n$ the division polynomial as defined above, then
$$ nP = (x - \frac{\Psi_{n-1} \Psi_{n+1}}{\Psi_n^2}, \frac{\Psi_{n+2} \Psi_{n-1}^2 - \Psi_{n-2} \Psi_{n+1}^2}{4y \Psi_n^3} )$$
\end{prop}

\subsection{Computing the number of points}
For an elliptic curve over $\mathbb{F}_q$ given by
$$ E: y^2 = x^3 + Ax + B $$
we want to compute the size of $\#E(\mathbb{F}_q)$, we know from before that
$$ \#E(\mathbb{F}_q) = q + 1 - t $$
where $t$ is the trace of the Frobenius as seen in section [referanse]. We know
that $t$ satisfies the Hasse bound namely
$$ |\#E(\mathbb{F}_q)-q-1|=|t| < 2\sqrt{q} $$
Let $S = \{3, 5, 7, 11, \ldots \, L \}$ be the set of odd primes $\leq L$ such
that the product is bigger than the Hasse interval
$$ N = \prod_{\ell \in S} \ell  > 4\sqrt{q} $$
If we can then calculate $t\, (mod \ell)$ for all $\ell \in S$ we can uniquely
determine $t\,(mod N)$ by invoking the Chinese remainder theorem,
which then by the Hasse bound is our Frobenius trace $t$.

The argument above is the gist of Schoof's algorithm, we will now look at
how to calculate $t\, (mod \ell)$. Let $\phi$ be the Frobenius endomorphism
resticted to $E[\ell]$ and let $q_\ell$, $\tau$ be $q$ and $t$ reduced modulo $\ell$
respectively. The computation of $\tau$ can then be done by checking if
$$ \phi^2(P) + q_\ell P = \tau \phi(P) $$
holds for $P \in E[\ell]$. To perform the addition on the left hand side of the
equality we need to distinguish the cases where the points are on a vertical line or not.
In other words we have to verify if for $P = (x,y) \in E[\ell]$ the following holds
$$ \phi^2 (P) = \pm q_\ell P $$
Noting that $-P = (x, -y)$ we write out the equality for the $x$-coordinates in terms of
division polynomials
$$ x^{q^2} = x - \frac{\Psi_{q_\ell-1} \Psi_{q_\ell+1}}{\Psi_{q_\ell}^2}(x,y) $$
Writing this out in terms of $f_n(x)$ and noting that for $n$ even we have
$\Psi_n(x,y) = y f_n(x)$, a calculation for $q_\ell$ even yields
\begin{eqnarray*}
 x^{q^2} &=& \frac{f_{q_\ell-1}(x) f_{q_\ell+1}(x)}{(f_{q_\ell} y)^2} \nonumber \\
	 &=& \frac{f_{q_\ell-1}(x) f_{q_\ell+1}(x)}{f_{q_\ell}^2 (x^3+Ax+B)} \nonumber \\
\end{eqnarray*}
The calculation for $q_\ell$ odd is similar and we get the equality

$$
x^{q^2} = \begin{cases}
           x - \frac{f_{q_\ell-1}(x) f_{q_\ell+1}(x)}{f_{q_\ell}^2 (x^3+Ax+B)} & \text{if } q_\ell \text{ is even} \\
	   x - \frac{f_{q_\ell-1}(x) f_{q_\ell+1}(x) (x^3+Ax+B)}{f_{q_\ell}^2(x)} & \text{if } q_\ell \text{ is odd} 
          \end{cases}
$$
We thus get two equations and we want to verify they have any solutions $P \in E[\ell]$. For
doing this we compute the following greatest common divisors
$$ gcd((x^{q^2} - x)f_{q_\ell}^2 (x^3+Ax+B)+f_{q_\ell-1}(x) f_{q_\ell+1}(x), f_\ell(x)) \quad (q_\ell \text{ even)}$$
$$ gcd((x^{q^2} - x)f_{q_\ell}^2(x)+f_{q_\ell-1}(x) f_{q_\ell+1}(x) (x^3+Ax+B), f_\ell(x)) \quad (q_\ell \text{ odd)}$$
We are now going to treat the rest in two cases, depending on the value from the above gcds.

\textbf{Case 1:} $gcd \neq 1$ meaning there exist a non-zero $\ell$-torsion point $P$ such that $\phi^2(P) = \pm q_\ell P$.
If $\phi^2 (P) = -q_\ell P$ we have that $\tau \phi(P) = 0$ but since $\phi(P) \neq 0$ we know that $\tau = 0$.
If $\phi^2(P) = q_\ell P$ we have that 
$$ 2 q_\ell P = \tau \phi(P) \Leftrightarrow \phi(P) = \frac{2 q_\ell}{\tau} $$
Substituting the last equality into $\phi^2(P) = q_\ell P$ we obtain
$$ \frac{4 q_\ell^2}{\tau^2} = q_\ell P \Leftrightarrow 4 q_\ell P = \tau^2 P $$
We thus obtain the congruence $\tau^2 \equiv 4q_\ell \, (mod \ell)$

\textbf{Case 2:} $gcd = 1$ so $\phi^2(P) \neq \pm q_\ell P$ meaning the two points are 
not equal nor are they on the same vertical line for any $\ell$-torsion point $P$. 
This enables us to do the addition $\phi^2(P) + q_\ell P$ using the appropriate addition formulas.
Recall that if $P = (x_1, y_1)$ and $Q = (x_2, y_2)$ are two points on $E$ with
$P \neq Q$ we have that their sum is given by $P+Q = (x_3, y_3)$ where
$$ \lambda = \frac{y_2 - y_1}{x_2 - x_1} $$
$$ x_3 = -x_1 - x_2 + \lambda^2 $$
$$ y_3 = -y_1 -\lambda(x_3 - x_1) $$
We can now write out the addition explicitly in terms of polynomials as follows
$$ \lambda = \frac{\Psi_{q_\ell+2} \Psi_{q_\ell-1}^2 - \Psi_{q_\ell-2}\Psi_{q_\ell+1}^2 - 4y^{q^2+1}\Psi_{q_\ell}^3}
		  {4\Psi_{q_\ell} y ((x-x^{q^2})\Psi_{q_\ell}^2 - \Psi_{q_\ell-1}\Psi_{q_\ell+1}} $$
$$\phi^2 (P) + qP = \left(-x^{q^2}-x+\frac{\Psi_{q_\ell-1}\Psi_{q_\ell+1}}{\Psi_{q_\ell}^2}+\lambda^2,
		     -y^{q^2}-\lambda\left(-2x^{q^2}-x+\frac{\Psi_{q_\ell-1}\Psi_{q_\ell+1}}{\Psi_{q_\ell}^2}\right)\right)$$
The right hand side is as before given by
$$ \tau\phi(P)=\left(x^q-\left(\frac{\Psi_{\tau+1}\Psi_{\tau-1}}{\Psi_\tau^2}\right)^q,\left(\frac{\Psi_{\tau+2}\Psi_{\tau-1}^2 - \Psi_{\tau-2}\Psi_{\tau+1}^2}{4y\Psi_\tau^3}\right)^q\right) $$

\subsection{Modular polynomials}
fixme
\subsection{Schoof-Elkies algorithm}
When doing the calculations in Schoof's algorithm we were working modulo the division
polynomials $\Psi(x,y)$ of degree $\ell^2-1$. Instead we can exploit some special primes
called \emph{Elkies primes} that enables us to work in a cyclic subgroup $C$ of $E[\ell]$.
Here $C$ will correspond to a $1$-dimensional eigenspace.

The frobenius endomorphism restricted to $E[\ell]$ satisfies the characteristic equation
$$ \phi^2 - \tau \phi + q_\ell = 0 $$
where $\tau$ and $q_\ell$ is as before. The roots of this equations are the eigenvalues
of $\phi | E[\ell]$ and they are given by
$$ \lambda_{1,2} = \frac{\tau \pm \sqrt{\tau^2 - 4 q_\ell}}{2} $$
If the discriminant $\tau^2 - 4 q_\ell$ is a square modulo $\ell$ we have that
$\lambda_{1,2} \in \mathbb{F}_q$.
\begin{mydef}
A prime $\ell$ such that $\tau^2 - 4 q_\ell$ is a square modulo $\ell$ is called
an \emph{Elkies prime}.
\end{mydef}
For primes of this type we obtain a factorization
$$ (\phi - \lambda_1)(\phi - \lambda_2) = 0 $$
so for an eigenvalue $\lambda$ we have that $\phi(P) = \lambda P$ for a point $P$. Thus
$P$ is the generator for a cyclic eigenspace $C \subset E[\ell]$ of order $\ell$
corresponding to $\lambda$.
Notice that we have an exact sequence of groups
$$ 0 \rightarrow C \rightarrow E \rightarrow E/C \rightarrow 0 $$
where the map $E \rightarrow E/C$ has cyclic kernel $C$ of order $\ell$.
Determining which primes are Elkies primes can be done by working with the
modular polynomials. From [ref til modulærpoly-teorem] we have that $\Phi_\ell(j(E),j(E/C)) = 0$,
so letting the isogeny $f: E \rightarrow E'$ have cyclic kernel $C$ we get an exact
sequence
$$ 0 \rightarrow C \rightarrow E \rightarrow E' \rightarrow 0 $$
which by a diagram chase yields $ E' \simeq E/C $. This argument gives us the following result
\begin{prop}
 $ \Phi_\ell(j(E), x) = 0$ for $x \in \mathbb{F}_q$ if and only if $\ell$ is an Elkies
prime.
\end{prop}
Figuring out if $\ell$ is an Elkies prime can thus be done fast by calculating
$$gcd(\Phi_\ell(j(E), x), x^q - x)$$
Now since we are working only with primes of this type we restrict ourself to working
in the subspace $C$ of order $\ell$. There is thus a factor $G_\ell(x)$ of the division polynomial
which has the $x$-coordinates of points in $C$ as roots. Since
similar points in $C$ of different sign are on the same vertical line we only include
unique points up to sign. In this way we get that the degree of $G_\ell(x)$ is $\frac{\ell-1}{2}$.

From the theory of eigenvalues we know that if $\lambda_1, \lambda_2$ are
eigenvalues of $\phi$ then $$tr(\phi) = \lambda_1 + \lambda_2$$
We also know using [referanse] that $$\lambda_1 \lambda_2 = det(\phi) = q $$
Using this we can recover the trace of $\phi$ by calculating one of the eigenvalues
$$ \tau \equiv \lambda + \frac{q}{\lambda} \quad (mod\, \ell) $$
To compute the eigenvalue $\lambda$ we can thus check which of the relations
$$ \phi(P) = (x^q, y^q) = \lambda P$$
holds on the eigenspace $C$, this mean we can work modulo $G_\ell(x)$. This enables us
to work in the much smaller ring $$\mathbb{F}_q[x,y]/(G(x), y^2 - x^3 - Ax - B) $$
and thus greatly improves Schoof's original approach.

The obstacle that remains is how we can possible calculate the factor 
$$ G_\ell(x) = \prod_{(x',y')\in C} (x-x')$$
of the division polynomialwhere the product is taken over all 
unique points $P = (x', y')$ up to sign. When calculating the gcd
$gcd(\Phi_\ell(j(E), x), x^q - x)$ we obtain a polynomial whos roots
(at most two) are the $j$-invariants of the $\ell$-isogenous curves
$\tilde{E} = E/C$ where $C$ is the eigenspace corresponding to $\lambda$.
The next theorem enables us to calculate an explicit formula for the Weierstrass
equation of $\tilde{E}$.

\begin{thm}
 Let $E$ be given by the equation
$$ E: y^2 = x^3 + Ax + B $$
with $j = j(E)$. Then the eqation for the $\ell$-isogenous curve $\tilde{E}$ with
$\tilde{j} = j(\tilde{E})$ is given by
$$ \tilde{E}: y^2 = x^3 + \bar{A}x + \bar{B} $$

$$\bar{A} = -\frac{\tilde{j}'^2}{48 \tilde{j}(\tilde{j} - 1728)} \quad
  \bar{B} = -\frac{\tilde{j}'^3}{864 \tilde{j}^2(\tilde{j} - 1728)} $$

And letting $$\Phi_{\ell, x} = \frac{\partial \Phi_\ell}{\partial x} \quad
              \Phi_{\ell, y} = \frac{\partial \Phi_\ell}{\partial y}$$
be the partial derivatives with respect to $x$ and $y$ respectively we have that
$$ \bar{j}' = -\frac{18 B \Phi_{\ell, x}(j, \bar{j})}{\ell A \Phi_{\ell, y}(j, \bar{j})} j $$
\end{thm}

