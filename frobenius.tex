\section{Frobnius and finite fields} \label{frob}
Throughout this section our fields $k$ will be finite, so let $char(k) = p$ for
a prime $p$. This means that $k = \mathbb{F}_{q}$ for some $q = p^r$.

\begin{mydef}
 The \emph{Frobenius} endomorphism is the $p^{th}$-power map
$$ \phi: k \rightarrow k $$
$$ x \mapsto x^p $$
which induces a map on curves as follows
$$ \phi: E(k) \rightarrow E^{\phi}(k) $$
$$ (x_0,\ldots , x_n) \mapsto (x_0^p, \ldots , x_n^p) $$
where $E^{\phi}$ is the curve $E$ with $\phi$ applied to its coefficients.
$$E: y^2 = x^3 + ax + b \quad E^{\phi}: y^2 = x^3 + \phi(a)x + \phi(b) $$
\end{mydef}

We can apply the Frobenius endomorphism $r$ times $$\phi^r(x) = x^{p^r} = x^q$$
And since every finite field of $q$ elements is the splitting field of $x^{q}-x$, it is in other words
the fixed points of the $q^{th}$ Frobenius endomorphism
$$ \phi^r(x) = x \iff x \in \mathbb{F}_q $$
The same is true for all intermediate fields of size $p^k$ with $0 < k \leq r$, so in general
we have that the $\phi^k$ fixes the elements of the field $\mathbb{F}_{p^k}$.
\begin{prop}
 The degree map
$$ deg: Hom(E_1, E_2) \rightarrow \mathbb{Z} $$
is a positive quadratic form.
\end{prop}
\begin{proof}
 Clearly $deg(f) = deg(-f)$. The only thing that takes a proof is the
bilinearity of the pairing
$$ Hom(E_1, E_2) \times Hom(E_1, E_2) \rightarrow \mathbb{Z}$$
$$ (\phi, \psi) \mapsto deg(\phi + \psi) - deg(\phi) - deg(\psi) $$
For this proof we will make extentive use of the dual isogeny, but first
notice that we have an injection of multiplication by $n$ maps
$$ [\quad]: \mathbb{Z} \rightarrow End(E_1) $$
A calculation then yields 
\begin{eqnarray*} 
 [\langle \phi,\psi \rangle] &=& [deg(\phi+\psi)]-[deg(\phi)]-[deg(\psi)] \nonumber \\
               &=& (\widehat{\phi+\psi})(\phi+\psi) - \widehat{\phi}\phi - \widehat{\psi}\psi \nonumber \\
	       &=& \widehat{\phi}\psi + \widehat{\psi}\phi
\end{eqnarray*}
The pairing is then shown to be linear in the first varible, the second variable is
similar.
\begin{eqnarray*}
 [\langle \phi_1+\phi_2, \psi \rangle] &=& \widehat{\psi}(\phi_1+\phi_2) + (\widehat{\phi_1+\phi_2})\psi \nonumber \\
			 &=& (\widehat{\psi}\phi_1+\widehat{\phi_1}\psi) + (\widehat{\psi}\phi_2 + \widehat{\phi_2}\psi) \nonumber \\
			 &=& [\langle \phi_1,\psi \rangle] + [\rangle \phi_2,\psi \rangle] 
\end{eqnarray*}
\end{proof}

\begin{thm} \label{frobkernel}
 Let $\phi$ be the $q^{th}$ frobenius map on $E/\mathbb{F}_q$. Then the map $1-\phi$ is seperable, and
$\#ker(1-\phi) = deg(1-\phi)$.
\end{thm}
\begin{proof}
  Recall from chapter \ref{diffsep} that a map $\psi$ is separable if and only if $\psi^*(\omega) \neq 0$,
where $\omega$ is the invariant differential. Using that the Frobenius $\phi$ is inseparable \cite{AEC}
we compute
\begin{eqnarray}
 (1-\phi)^*(\omega) &=& [1]^*\omega - \phi^*(\omega) \nonumber \\
		    &=& \omega - 0 \nonumber \\
		    &=& \omega \nonumber
\end{eqnarray}
thus $(1-\phi)^*(\omega) = 0$ if and only if $\omega = 0$, but the invariant differential is non-zero
so $(1-\phi)^*(\omega) \neq 0$ which means $1-\phi$ is separable.
\end{proof}

\begin{lemma}
 \textbf{(Cauchy-Schwartz inequality)}. Let $A$ be an abelian group and
$$ d: A \rightarrow \mathbb{Z} $$
a positive definite quadratic form. Then for all $\psi, \phi \in A$ the following holds
$$ |d(\psi-\phi)-d(\phi)-d(\psi)| \leq 2 \sqrt{d(\phi)d(\psi)} $$
\end{lemma}
\begin{proof}
 Let $\psi, \phi \in A$. From the definition of a quadratic form there is a bilinear pairing
$$ L(\psi, \phi) = d(\psi-\phi) - d(\psi) - d(\phi) $$
Using this definition, the fact that $d$ is positive definite and letting $m,n \in \mathbb{Z}$ where
$m = -L(\psi, \phi)$ and $n = 2d(\psi)$ we calculate

\begin{eqnarray}
 0 \leq d(m\psi - n\phi) &=& d(m\psi) + L(m\psi, n\phi) + d(n\phi) \nonumber \\
			 &=& m^2 d(\psi) + mnL(\psi,\phi) + n^2 d(\phi) \nonumber \\
			 &=& d(\psi) \left( 4d(\psi)d(\phi)-L(\psi, \phi)^2 \right) \nonumber 
\end{eqnarray}

where on the last line we make the substitution. If $d(\psi)=0$ the inequality is trivial, if
$d(\psi) \neq 0$ then we divide it out and obtain our result
$$L(\psi, \phi)^2 \leq 4d(\psi)d(\phi) $$
\end{proof}

\begin{thm}
 \textbf{(Hasse's theorem)}. Let $E$ be an elliptic curve over a finite field $k$ with $q$ elements, then
$$ |\#E(k) - q - 1| \leq 2\sqrt{q} $$
\end{thm}
\begin{proof}
 We let $\phi_q: E \rightarrow E$ be the $q^{th}$ Frobenius endomorphism on $E$ given by 
$(x,y) \mapsto (x^q, y^q)$. Recall that $\phi_q$ fixes our field of $q$ elements, thus
$$ P \in E(k) \quad \iff \quad \phi_q(P) = P$$
Writing out the right hand side of the implication we see that
$$ 0 = P - \phi_q(P) = (1 - \phi_q)(P) $$
which enables us to count the number of points in $E(k)$ by counting the number of points in the kernel
of the seperable map $1-\phi_q$. Recall from before that the number of points in the kernel is equal
to the degree of the seperable map
$$ \#E(k) = \# ker(1-\phi_q) = deg(1-\phi_q) $$
We have shown in that the degree map on $End(E)$ is a positive definite quadratic form, so
by using the inequality from the previous theorem we calculate
$$|deg(1-\phi_q) - deg(\phi_q) - deg(1)| = |\#E(k) - q - 1| \leq 2\sqrt{deg(\phi_q)} = 2\sqrt{q}$$
\end{proof}

\begin{prop} 
 If $\psi \in End(E)$ then $det(\psi_\ell) = deg(\psi)$, where $\psi_\ell$ is a $2\times2$ matrix acting
on the Tate module $T_\ell(E)$.
\label{detdeg}
\end{prop}
\begin{proof}
 We fix a basis $v_1,v_2 \in \mathbb{Z}_\ell \times \mathbb{Z}_\ell$ for $T_\ell(E)$ and denote the matrix
associated to this basis by
$$ \psi_\ell = \begin{pmatrix} a & b \\ c & d \end{pmatrix} $$
We now calculate by relying heavily on the $\ell$-adic Weil pairing,
$e: T_\ell(E) \times T_\ell(E) \rightarrow T_\ell(\mu)$.
\begin{eqnarray}
 e(v_1, v_2)^{deg(\psi)} &=& e([deg \psi]v_1, v_2) \nonumber \\
			 &=& e(\psi_\ell \widehat{\psi_\ell} v_1, v_2) \nonumber \\
			 &=& e(\psi_\ell v_1, \psi_\ell v_2) \nonumber \\
			 &=& e(a v_1 + c v_2, b v_1 + d v_2) \nonumber \\
			 &=& e(a v_1, d v_2) e(c v_2, b v_1) \nonumber \\
			 &=& e(a v_1, d v_2) e(b v_1, c v_2)^{-1} \nonumber \\
			 &=& e(v_1, v_2)^{ad} e(v_1, v_2)^{-bc} \nonumber \\
			 &=& e(v_1, v_2)^{ad - bc} \nonumber \\
			 &=& e(v_1, v_2)^{det \psi_\ell} \nonumber
\end{eqnarray}
Since the pairing is non-degenerate we obtain $deg(\psi) = det(\psi_\ell)$.
\end{proof}

Writing out the determinant of $1-A$ for any matrix $A$ we get
$$ \begin{vmatrix} 1-a & -b \\ -c & 1-d \end{vmatrix} = 1-(a+d)+ad-bc = 1-tr(A)+det(A) $$
so we see that $tr(\psi_\ell) = 1 + det(\psi_\ell) - det(1-\psi_\ell)$. Using the previous theorem we
get $$tr(\psi_\ell) = 1 + deg(\psi_\ell) - deg(1-\psi_\ell)$$ by substituting with the $q^{th}$ 
Frobenius endomorphism on $T_\ell(E)$ and setting $\tau = tr(\phi_q)$ we get
$$\#E(k) = 1 + q - \tau$$
where we know from Hasse's theorem that $|\tau| \leq 2\sqrt{q}$.

The next proposition will be used in chapter \ref{satoh}, it is easy to prove and gives a nice
expression of the Frobenius trace in terms of the dual isogeny.

\begin{prop}
 Let $\phi: E \rightarrow E$ be the $q^{th}$ Frobenius endomorphism and $\widehat{\phi}$ its dual, then
the following holds
$$ t = tr(\phi) = \phi + \widehat{\phi}$$
\end{prop}
\begin{proof}
 Recall that $1-\phi$ is seperable, so $$(1-\phi)(\widehat{1-\phi}) = deg(1-\phi) = \#ker(1-\phi) = \#E(k)$$
Expanding the product on the left we get
\begin{eqnarray}
 (1-\phi)(\widehat{1-\phi}) &=& (1-\phi)(1-\widehat{\phi}) \nonumber \\
			    &=& 1 - (\phi + \widehat{\phi}) + \phi\widehat{\phi} \nonumber \\
			    &=& 1 - (\phi + \widehat{\phi}) + q \nonumber
\end{eqnarray}
From before we had that $\#E(k) = q + 1 - t$ and we just calculated that $\#E(k) = q + 1 - (\phi +\widehat{\phi})$ so
the result follows.
\end{proof}

