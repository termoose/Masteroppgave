\documentclass[a4paper,10pt]{amsart}
\usepackage{amsthm}

%opening
\title{Test}
\author{Ole Andre Birkedal}

\begin{document}
\newtheorem{mydef}{Definition}
\newtheorem{ex}{Example}
\newtheorem{prop}{Proposition}

\maketitle
%\tableofcontents
\begin{abstract}
hei hei
\end{abstract}

\section{Algebraic geometry}
In this section we define the fundamental objects in algebraic geometry and state
some facts about their structure. We will then move on to the theory of
curves and Weil divisors.

\begin{mydef}
\emph{Projective n-space} over a field $k$ denoted $\mathbb{P}^n$ is the set 
of all $(n+1)$-tuples $$(x_0,\ldots,x_n)\in\mathbb{A}^{n+1}$$
modulo the equivalence relation given by $(x_0,\ldots,x_n)\sim(y_0,\ldots,y_n)$ 
if there exists $\lambda\in k$ such that $x_i=\lambda y_i$.
The equivalence class $\{(x_0,\ldots,x_n)\}$ is denoted $[x_0,\ldots,x_n]$.
Here $\mathbb{A}^n = \{ (x_1,\ldots,x_n) : x_i \in \bar{k} \} $ is the affine $n$-space.
\end{mydef}

Let $Gal(\bar{k}/k)$ be the galois group of $\bar{k}/k$. This group acts on
$\mathbb{A}^n$, such that when $\sigma \in Gal(\bar{k}/k)$ and $P\in \mathbb{A}^n$
we define $\sigma(P) = (\sigma(x_1),\ldots,\sigma(x_n))$. Now we define
the set of $k$-rational points in $\mathbb{A}^n$ to be those fixed under action by
the galois group
$$ \mathbb{A}^n(k) = \{ P \in \mathbb{A}^n : \sigma(P) = P\, \forall\, \sigma \in
Gal(\bar{k}/k) \} $$

Similarly we define the set of $k$-rational points in $\mathbb{P}^n$ to be
$$ \mathbb{P}^n(k) = \{ P \in \mathbb{P}^n : \sigma(P) = P\, \forall\, \sigma \in 
Gal(\bar{k}/k) \} $$

\begin{mydef}
 A polynomial $f\in\bar{k}[X]$ is said to be \emph{homogeneous of degree $d$} if for all
$\lambda\in\bar{k}$ we have.
$$f(\lambda x_0,\ldots,\lambda x_n) = \lambda^d f(x_0,\ldots,x_n)$$
Furthermore an ideal $I\subseteq\bar{k}[X]$ is said to be homogeneous if it is generated
by homogeneous polynomials.
\end{mydef}

\begin{mydef}
 A \emph{projective algebraic set} is of the form
$$ V_I = \{ P\in \mathbb{P}^n : f(P) = 0\, \forall homogeneous\, f\in I \} $$
Given such a set $V$ we associate to it an ideal $I(V) \in \bar{k}[X]$ generated by
$$ \{f\in\bar{k} : f\, homogeneous\, and\, f(P)=0\, \forall P\in V \} $$
\end{mydef}

\begin{mydef}
 A projective algebraic set is called a \emph{projective variety} if the homogeneous
ideal defined above is a prime ideal in $\bar{k}[x]$.
\end{mydef}

\begin{mydef}
 Let $V/k$ be a projective variety (i.e. V defined over $k$), then the projective coordinate
ring of $V/k$ is defined by
$$ k[V] = \frac{k[x]}{I(V/k)}$$
Note that since $I(V/k)$ is a prime ideal, the coordinate ring is an integral domain.
This enables us to form the quotient field of $k[V]$ which we denote $k(V)$, and it is called
the \emph{function field} of $V$.
\end{mydef}

A rather interesting ideal to keep in mind is given by
$$ M_p = \{ f\in \bar{k}[V] : f(P)=0 \} $$
This is a maximal ideal because the map $\phi: \bar{k}[V] \rightarrow \bar{k}$ given by
$ f \mapsto f(P) $ has kernel exactly $M_p$. It is clearly onto, so it induces an
isomorphism $$\tilde{\phi}: \bar{k}[V]/M_p \rightarrow \bar{k} $$

\begin{mydef}
 The \emph{localization of $\bar{k}[V]$ at $M_p$} is given by
$$ \bar{k}[V]_P = \{ h \in \bar{k}[V] : h = f/g\, f,g\in \bar{k}[V]\, and\, g(P)=0 \} $$
The functions in $\bar{k}[V]_P$ are all defined at $P$.
\end{mydef}

\begin{ex}
 If $V$ is a variety given by a single non-constant polynomial equation
$$f(x_1,\ldots,x_n) = 0$$ 
then the dimension of the variety $dim(V)$ is $n-1$. The (projective) varieties
we will study are called \emph{elliptic curvevs} and are
given by polynomial equations
$$E: y^2 = x^3+ax+b$$
They correspond to polynomials of the form $f(x,y) = x^3+ax+b-y^2$ so $dim(E)=1$.
We say curves are projective varieties of dimension $1$.
\end{ex}


The objects we will be working on are projective varieties, but they are not
very interesting unless we define maps between them.

\begin{mydef}
 Let $V_1$ and $V_2$ be projective varieties, a \emph{rational map} $\phi: V_1 \rightarrow V_2$
is a set of maps $\{\phi_0,\ldots,\phi_n\}$ with $\phi_i \in \bar{k}(V_1)$ such that for every
$P\in V_1$ we define
$$\phi(P) = [\phi_0(P),\ldots,\phi_n(P)] \in V_2$$
Such a rational map is called a \emph{morphism} if it is defined at every point $P$.
\end{mydef}

The varieties and the morphisms between them make up a category, so our next
definition of an isomorphism will be the general one found in category theory.

\begin{mydef}
 Two varieties $V$ and $W$ are \emph{isomorphic} denoted $V\simeq W$
if there exist morphisms $\phi: V \rightarrow W$ and $\psi: W \rightarrow V$ such that
$\phi \psi = 1_W$ and $\psi \phi = 1_V$.
If the rational functions $\psi$ and $\phi$ are defined over $k$ we say that $V$ and $W$
are isomorphic over $k$. If not, they are isomorphic over some field extension of $k$
(i.e. $\bar{k}$).
\end{mydef}

Recall that curves are projective varieties of dimension one. Even more special
are elliptic curves, which are curves with \emph{genus} equal to 1. This will
be introduced later on. These are in practise the only curves we will be working with.

\begin{mydef}
 Let $C$ be a curve and $P\in C$ a non-singular point on the curve. A valuation on
$\bar{k}[C]_P$ is given by
$$ ord_P : \bar{k}[C]_P \rightarrow \{ 0, 1, 2, \ldots \} \cup \{ \infty \} $$
$$ ord_P(f) = max \{ d\in \mathbb{Z} : f\in M_P^d \} $$
This is called \emph{the order of $f$ at $P$}.
Letting $ord_P(f/g) = ord_P(f) - ord_P(g)$ we can extend the definition to the entire
quotient ring $\bar{k}(C)$
$$ ord_P: \bar{k}(C) \rightarrow \mathbb{Z}\cup \{\infty \} $$
\end{mydef}

The definition of order agrees with the one found in complex analysis.
If $ord_P(f) < 0$ f has a pole at $P$ and we write $f(P)=\infty$. 
If $ord_P(f) \ge 0$ f has a zero and is defined at $P$, so $f(P)$ can be calculated. 

\begin{prop}
 Let $C$ be a smooth curve. If $f\in \bar{k}(C)$ is not the constant function, then
$f$ has finitely many poles and zeros.
\label{prop:1}
\end{prop}
\begin{proof}
 FIXME. Prop 1.2 AEC.
\end{proof}

\begin{mydef}
 The \emph{divisor group} of a curve $C$ is the free abelian group generated by
points of $C$, denoted $Div(C)$. A divisor $D\in Div(C)$ is of the form
$$ D = \sum_{P\in C} n_P(P)$$
with $n_P\in\mathbb{Z}$ and $n_P = 0$ for almost all $P$.
\end{mydef}

With this in mind we can define the degree of a divisor as the sum of its 
coefficients. We also define the sum of a divisor as the sum in the group $E(\bar{k})$, so
$$ deg(D) = deg(\sum_{P\in C} n_P(P)) = \sum_{P\in C} n_P \in \mathbb{Z}$$
$$ sum(D) = sum(\sum_{P\in C} n_P(P)) = \sum_{P\in C} n_P P \in E(\bar{k})$$

Now let $C$ be a smooth curve and $f\in \bar{k}(C)$ a non-zero function. Since $f$
has finitely many poles and zeros (Prop. \ref{prop:1}) we can define the divisor of a
function as
$$ div(f) = \sum_{P\in C} ord_P(f)(P) $$
Note that $ord_P$ is a valution we have $ord_P(fg) = ord_P(f)+ord_P(g)$
for non-zero $f,g\in \bar{k}(C)$. Thus we get a group homomorphism
$$ div: \bar{k}(C)^* \rightarrow Div(C)$$



\end{document}
