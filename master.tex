\documentclass[a4paper,10pt]{amsart}
\usepackage{amsthm,mathrsfs}

%opening
\title{Counting points on elliptic curves}
\author{Ole Andre Birkedal}

\begin{document}
\newtheorem{thm}{Theorem}
\newtheorem{mydef}{Definition}
\newtheorem{ex}{Example}
\newtheorem{prop}{Proposition}
\newtheorem{lemma}{Lemma}

\begin{abstract}
hei hei
\end{abstract}

\maketitle
%\tableofcontents

%
\section{Algebraic curves}
In this section we define the fundamental objects in algebraic geometry and state
some facts about their structure. We will then move on to the theory of
curves and Weil divisors. We will closely be following \cite{AEC} with some aid from
\cite{Fulton}.

Given a field $k$ we write $\mathbb{A}^n = \left\{ (x_1,\ldots,x_n) : x_i\in \bar{k}\right\}$ for the
affine $n$-space, where $\bar{k}$ denotes the algebraic closure of the field $k$.

\begin{mydef}
\emph{Projective $n$-space} over a field $k$ denoted $\mathbb{P}^n$ is the set 
of all $(n+1)$-tuples $$(x_0,\ldots,x_n)\in\mathbb{A}^{n+1}$$
modulo the equivalence relation given by $(x_0,\ldots,x_n)\sim(y_0,\ldots,y_n)$ 
if there exists $\lambda\in \bar{k}$ such that $x_i=\lambda y_i$.
The equivalence class containing $(x_0, \ldots, x_n)$ is denoted $[x_0,\ldots,x_n]$.
\end{mydef}
Elements of the real projective $1$-space can be identified with lines through the origin, that is the
$1$-dimensional subspaces of $k^2$.

Let $Gal(\bar{k}/k)$ be the Galois group of $\bar{k}/k$. This group acts on
$\mathbb{A}^n$ in the following way: given $\sigma \in Gal(\bar{k}/k)$ and $P\in \mathbb{A}^n$
we define $\sigma(P) = (\sigma(x_1),\ldots,\sigma(x_n))$. Now we can define
the set of $k$-rational points in $\mathbb{A}^n$ to be those fixed under action by
the Galois group
$$ \mathbb{A}^n(k) = \left\{ P \in \mathbb{A}^n : \sigma(P) = P\quad \forall\, \sigma \in
Gal(\bar{k}/k) \right\}. $$

Similarly it can be shown that the set of $k$-rational points in $\mathbb{P}^n$ are
$$ \mathbb{P}^n(k) = \left\{ P \in \mathbb{P}^n : \sigma(P) = P\quad \forall\, \sigma \in 
Gal(\bar{k}/k) \right\}. $$

\begin{mydef}
 A polynomial $f\in\bar{k}[x_0,\ldots,x_n]$ is said to be \emph{homogeneous of degree $d$} if for all
$\lambda\in\bar{k}$ we have.
$$f(\lambda x_0,\ldots,\lambda x_n) = \lambda^d f(x_0,\ldots,x_n).$$
Furthermore an ideal $I\subseteq\bar{k}[X]$ is said to be \emph{homogeneous} if it is generated
by homogeneous polynomials.
\end{mydef}

\begin{mydef}
 A \emph{projective algebraic set} is of the form
$$ V_I = \left\{ P\in \mathbb{P}^n : f(P) = 0\, \forall homogeneous\, f\in I \right\}.$$
Given such a set $V$ we associate to it an ideal $I(V) \in \bar{k}[x_0,\ldots, x_n]$ generated by
$$ \left\{f\in\bar{k} : f\, homogeneous\, and\, f(P)=0\, \forall P\in V \right\}.$$
\end{mydef}

\begin{mydef}
 A projective algebraic set $V$ is called a \emph{projective variety} if the homogeneous
ideal defined above is a prime ideal in $\bar{k}[x_0,\ldots,x_n]$.
\end{mydef}

We say that the variety $V$ is \emph{defined over $k$}, denoted $V/k$, if its associated ideal $I(V)$
can be generated by polynomials in $k[x_0,\ldots,x_n]$.

\begin{mydef}
 Let $V/k$ be a projective variety, then the projective coordinate ring of $V/k$ is defined by
$$ k[V] = \frac{k[x]}{I(V/k)}.$$
Note that since $I(V/k)$ is a prime ideal, the coordinate ring is an integral domain.
This enables us to form the quotient field of $k[V]$ which we denote $k(V)$, and it is called
the \emph{function field} of $V$.
\end{mydef}

A rather interesting ideal to keep in mind is given by
$$ M_P = \left\{ f\in \bar{k}[V] : f(P)=0 \right\}. $$
This is a maximal ideal because the map $\phi: \bar{k}[V] \rightarrow \bar{k}$ given by
$ f \mapsto f(P) $ has kernel exactly $M_P$. It is clearly onto, so it induces an
isomorphism $$\tilde{\phi}: \bar{k}[V]/M_P \rightarrow \bar{k}. $$

\begin{mydef}
 The \emph{localization of $\bar{k}[V]$ at $M_P$} is given by
$$ \bar{k}[V]_P = \left\{ h \in \bar{k}[V] : h = f/g\, f,g\in \bar{k}[V]\, and\, g(P)=0 \right\}. $$
The functions in $\bar{k}[V]_P$ are all defined at $P$.
\end{mydef}

For more information about localization in commutative rings I refer to \cite{Hideyuki}.

%\begin{ex}
% If $V$ is a variety given by a single non-constant polynomial equation
%$$f(x_1,\ldots,x_n) = 0$$ 
%then the dimension of the variety $dim(V)$ is $n-1$. The (projective) varieties
%we will study are called \emph{elliptic curves} and are
%given by polynomial equations
%$$E: y^2 = x^3+ax+b$$
%They correspond to polynomials of the form $f(x,y) = x^3+ax+b-y^2$ so $dim(E)=1$.
%We say curves are projective varieties of dimension $1$.
%\end{ex}

\begin{mydef}
 Let $V$ be a variety, then the \emph{dimension} of $V$ is the transcendence degree of $\bar{k}(V)$ over
$\bar{k}$. We denote this value by $dim(V)$.
\end{mydef}

Given the above definition we have in particular that the transcendence degree of $\bar{k}(x,y)$ over
$\bar{k}$ is $2$, because $x$ and $y$ are two (independent) transcendental variables.

\begin{ex}
 Let $V$ be the variety given as
$$ V: y^2 = x^3 + ax + b. $$
This corresponds to a polynomial $f(x,y) = x^3+ax+b-y^2 \in \bar{k}[x,y]$. Since there
is imposed a relation between $x$ and $y$ we  have that the transcendence degree is $1$ and
$dim(V)=1$. The varieties of dimension $1$ are called \emph{curves}, and is what we'll be working with.
\end{ex}
We now have the objects, the next logical step is to define maps between the varieties.
\begin{mydef}
 Let $V_1$ and $V_2$ be projective varieties, a \emph{rational map} $\phi: V_1 \rightarrow V_2$
is a set of maps $\left\{\phi_0,\ldots,\phi_n\right\}$ with $\phi_i \in \bar{k}(V_1)$ such that for every
$P\in V_1$ we define
$$\phi(P) = [\phi_0(P),\ldots,\phi_n(P)] \in V_2.$$
Such a rational map is called a \emph{morphism} if it is defined at every point $P$.
\end{mydef}

The varieties and the morphisms between them make up a category, so our next
definition of an isomorphism will be the general one found in category theory.

\begin{mydef}
 Two varieties $V$ and $W$ are \emph{isomorphic} denoted $V\simeq W$
if there exist morphisms $\phi: V \rightarrow W$ and $\psi: W \rightarrow V$ such that
$\phi \psi = 1_W$ and $\psi \phi = 1_V$.
If the rational functions $\psi$ and $\phi$ are defined over $k$ we say that $V$ and $W$
are isomorphic over $k$. If not, they are isomorphic over some field extension of $k$
(i.e. $\bar{k}$).
\end{mydef}

Recall that curves are projective varieties of dimension one. Even more special
are elliptic curves, which are curves with \emph{genus} equal to 1. This will
be introduced later on. These are in practise the only curves we will be working with.

Composition of points on an elliptic curve can be done in the following way:
let $P,Q\in E$ and $l$ the line connecting them. We let $R$ be the third
point that $l$ intersects, then composition denoted $P+Q$ is the mirror
point of $R$ (i.e. $-R$). See Figure \ref{fig1}.

\begin{prop}
  An elliptic curve $E$ is an abelian group with the group operation as described above.
The identity element is denoted $O$.
\end{prop}
For a proof of the above proposition i refer to \cite{AEC}.

Out fields $k$ shall never be of characteristic $2$ or $3$, this enables us to assume
that every elliptic cure is given by a Weierstrass equation of the form
$$ E: y^2 = x^3 + ax + b$$
with $a,b\in k$ \cite{AEC}.

\begin{ex}
 As an example of point addition we consider
$$E: y^2 = x^3-5x+13$$
as in Figure \ref{fig1}. We have that $P=(-3,1)$ and $Q=(1,3)$. By the composition law
as described above we have that $P+Q=(\frac{9}{4}, \frac{-29}{8})$.
\end{ex}



\begin{figure}\label{fig1}
  \centering
  \includegraphics[width=80mm]{ellipticplot}
 \caption{The elliptic curve $y^2=x^3-5x+13$}
\end{figure}

\begin{mydef}
 Let $C$ be a curve defined by the polynomial equation
$$f(x,y) = 0$$
and $P=(x_0,y_0) \in C$ a point on the curve. Then $P$ is \emph{singular} if and only if all
partial derivatives vanish at $P$
$$\frac{\partial}{\partial x}f(P) = \frac{\partial}{\partial y}f(P) = 0.$$
\end{mydef}

This is in fact the implicit function theorem at work, saying that there is no way to
represent the curve as the graph of a function of one variable near $P$.
In addition we say that a curve $C$ is \emph{smooth} if it has no singular points.

\begin{mydef}
 Let $C$ be a curve and $P\in C$ a non-singular point on the curve. The \emph{valuation} on
$\bar{k}[C]_P$ is given by
$$ ord_P : \bar{k}[C]_P \rightarrow \left\{ 0, 1, 2, \ldots \right\} \cup \left\{ \infty \right\} $$
$$ ord_P(f) = max \left\{ d\in \mathbb{Z} : f\in M_P^d \right\} $$
This is called \emph{the order of $f$ at $P$}.
Letting $ord_P(f/g) = ord_P(f) - ord_P(g)$ we can extend the definition to the entire
quotient ring $\bar{k}(C)$
$$ ord_P: \bar{k}(C) \rightarrow \mathbb{Z}\cup \{\infty \}.$$
\end{mydef}

The definition of order agrees with the one found in complex analysis.
If $ord_P(f) < 0$ then $f$ has a pole at $P$, similarly if $ord_P(f) \ge 0$ then $f$ 
has a zero and is defined at $P$.

\begin{prop}
 Let $C$ be a smooth curve. If $f\in \bar{k}(C)$ is not the constant function, then
$f$ has finitely many poles and zeros.
\label{prop:1}
\end{prop}

\begin{mydef}
 The \emph{divisor group} of a curve $C$ is the free abelian group generated by
points of $C$, denoted $Div(C)$. A divisor $D\in Div(C)$ is of the form
$$ D = \sum_{P\in C} n_P(P)$$
with $n_P\in\mathbb{Z}$ and $n_P = 0$ for all but finitely many $P$.
\end{mydef}

With this in mind we can define the degree of a divisor as the sum of its 
coefficients. We also define the sum of a divisor as the sum in the group $E(\bar{k})$, so
$$ \deg(D) = \deg\left(\sum_{P\in C} n_P(P)\right) = \sum_{P\in C} n_P \in \mathbb{Z}$$
$$ sum(D) = sum\left(\sum_{P\in C} n_P(P)\right) = \sum_{P\in C} n_P P \in E(\bar{k}).$$

These functions enable us to define the subgroup of divisors of degree zero,
$Div^0(C) \subset Div(C)$, so $Div^0(C) = \left\{ D\in Div(C) : \deg D  = 0 \right\}.$

Now let $C$ be a smooth curve and $f\in \bar{k}(C)$ a non-zero function. Since $f$
has finitely many poles and zeros (Prop. \ref{prop:1}) we can define the divisor of a
function as
$$ div(f) = \sum_{P\in C} ord_P(f)(P) $$
Note that $ord_P$ is a valuation we have $ord_P(fg) = ord_P(f)+ord_P(g)$
for non-zero $f,g\in \bar{k}(C)$. Thus we get a group homomorphism
$$ div: \bar{k}(C)^* \rightarrow Div(C).$$

\begin{mydef}
 The \emph{principal divisors} of $C$ are the divisors of the form
$ D = div(f) $ for some non-zero $f\in \bar{k}(C)$. This is exactly
the image of the function $div$ and we denote this set by $Prin(C)$.
\end{mydef}

Note that since divisors of rational functions have the same number of poles
and zeros (when counted correctly), we have $\deg(div(f)) = 0$.

Two divisors are said to be \emph{equivalent} denoted $D_1 \sim D_2$ if
their difference is a principal divisor, $D_1 - D_2 = div(f)$ for some $f\in \bar{k}(C)$.
We say that a divisor $D$ is
\emph{positive} $\sum n_P(P)=D \geq 0$ if $n_P \geq 0$ for every $P\in C$. Furthermore
we can put a partial ordering on on $Div(C)$ writing $D_1 \geq D_2$ to indicate that $D_1 - D_2$ is positive.

\begin{mydef}
 Let $C$ be a curve. The \emph{Picard group of C} is the quotient $Div(C)/Prin(C)$ and is denoted $Pic(C)$.
Note that since $Prin(C) \subseteq Div^0(C)$ we define $Pic^0(C) = Div^0(C)/Prin(C)$ which is the
\emph{degree $0$ part} of the Picard group.
\end{mydef}


\begin{ex}
 Inequalities can easily summarize some key properties of a function. So instead of
saying $f \in \bar{k}(C)$ is regular everywhere except at $P$ and $Q$, where it has a
pole and a root of order $m$ and $n$ respectively, we could write
$$ div(f) \geq -m(P)+n(Q).$$
\end{ex}



The last example motivates our next definition, where we collect all functions
which satisfy some inequality. This turns out to make up a finite dimensional
$\bar{k}$-vector space.

\begin{mydef}
 Let $D \in Div(C)$ be a divisor, and we define the set of functions
$$ \mathscr{L}(D) = \left\{ f\in \bar{k}(C) : div(f) \geq -D \right\} \cup \left\{ 0\right\}. $$
\end{mydef}

This vector space can be seen to be finite by the next proposition, a proof of
which can be found in \cite{Fulton}.

\begin{prop}
 $\mathscr{L}(D)$ is a finite dimensional $\bar{k}$-vector space, and we denote
its dimension by
$$ \ell(D) = dim_{\bar{k}} \mathscr{L}(D). $$
\end{prop}

Next we introduce differential forms on our curves, these will be useful for different purposes
as described below. In addition they will help us state the Riemann-Roch theorem and a definition
of the genus $g$.

\begin{mydef}
 The space of differential forms on a curve $C$ is a $\bar{k}(C)$-vector space denoted $\Omega_C$
generated by symbols subject to the relations known from analysis. 
For $x, y \in \bar{k}(C)$ and $a \in \bar{k}$
\begin{enumerate}
  \item $d(x+y) = dx + dy$
  \item $d(xy) = xdy + ydx$
  \item $da = 0$
\end{enumerate}
Let $f_i \in \bar{k}(C)$ and $dx_i$ be the symbols as defined above, a general
element $\omega \in \Omega_C$ is of the form
$$ \omega = \sum f_i dx_i.$$
\end{mydef}

It can be shown that if $t\in \bar{k}(C)$ is the uniformizer \cite{Fulton} then
$\omega = f dt$ for some $f\in \bar{k}(C)$ and we define
$$ord_P(\omega) = ord_P(f).$$

The divisor of a differential is given by
$$ div(\omega) = \sum_{P\in C} ord_P(\omega)(P) \in Div(C).$$
Divisors in the image of the map $div: \Omega_C \rightarrow Div(C)$ are called
\emph{canonical divisors}. They will play a role in the next theorem which will
serve as an important tool for calculating the dimension of the vector space
$\mathscr{L}(D)$, which will be crucial in establishing an important isomorphism.

\begin{thm}
 \textbf{(Riemann-Roch)}
  Let $C$ be a smooth curve and $K_C$ a canonical divisor on $C$. Then for
any $D \in Div(C)$ we have
$$ \ell(D) - \ell(K_C - D) = \deg D - g + 1 $$
where $g \geq 0$ is called the \emph{genus} of the curve $C$.
\end{thm}

 A proof would be well outside the scope of this paper, but a classical proof based
on Noether's reduction lemma can be found in \cite{Fulton}. Almost directly from the
above theorem follows a nice corollary, its short proof can be found in \cite{AEC}.

\begin{cor} \label{rrcor}
Let $K_C$ be a canonical divisor, then
 \begin{enumerate}
  \item[a)] $\ell(K_C) = g$
  \item[b)] $\deg K_C = 2g - 2$
  \item[c)] $\deg D > 2g - 2 \implies \ell(D) = \deg D - g + 1$ 
 \end{enumerate}
\end{cor}

Given a non-constant map of curves $\phi: C_1 \rightarrow C_2$, we have an induced map
on function fields $$\phi^*: K(C_2) \rightarrow K(C_1)$$
$$f \mapsto f \phi.$$
From this again we get an induced
map on differential forms
$$ \phi^*: \Omega_{C_2} \rightarrow \Omega_{C_1} $$
$$ \phi^*\left(\sum f_i dx_i\right) = \sum (\phi^* f_i) d(\phi^* x_i).$$
\label{diff}

\begin{mydef} 
 Let $\phi: C_1 \rightarrow C_2$ be a map of curves and $\phi^*$ its induced map on
function fields. We then say that $\phi$ is a \emph{separable map} if
$K(C_1)/\phi^* K(C_2)$ is a separable extension.
\end{mydef}

Similarly, the degree of $\phi$ (as above) is the degree of the associated field extension.
Recall that a field extension is separable if and only if the derivative of the minimal
polynomial for each element is non-zero. This fact is the motivation for our next result,
which gives a useful criterion for determining when a map is separable.

\begin{prop} \label{diffsep}
 Let $\phi: C_1 \rightarrow C_2$ be a map of curves, then $\phi$ is separable if and only if
the induced map $\phi^*: \Omega_{C_2} \rightarrow \Omega_{C_1} $ is non-zero.
\end{prop}

The next result gives us the key property that we need in separable maps, important for
point counting.
\begin{thm} \label{kerdeg}
 If $\phi$ is a separable map then
$$\#\ker \phi = \deg \phi.$$
\end{thm}

Before leaving the realm of differentials for a while we introduce a special differential.
\begin{mydef}
The \emph{invariant differential} on $E: y^2 = x^3 + ax + b$ is given by
$$\omega = \frac{dx}{\frac{d}{dy}\left(y^2 - x^3  - ax - b\right)} = \frac{dx}{2y}.$$
\end{mydef}
This is a \emph{holomorphic} differential, having no poles or zeros \cite{AEC}. The name comes from it
being invariant under the translation isomorphism
$$ t_Q: E \rightarrow E$$
$$ P \mapsto P + Q$$
as shown in \cite{AEC}. We note that since $t_Q^*(\omega) = \omega$ we have that
$t_Q^*(k\omega) = k t_Q^*(\omega) = k\omega$ for any integer $k$. Especially we have that
$2\omega = \frac{dx}{y}$ is also invariant under translation, this differential will be used
in Section \ref{satoh} \label{invariant}.

\begin{prop} \label{3.3}
 Let $C$ be a curve of genus $1$ (think elliptic curve), and let $P,Q\in C$ be points on the curve and
$(P), (Q)$ their corresponding divisors. Then we have that
$$ (P) \sim (Q) \iff P = Q.$$
\end{prop}
\begin{proof}
 We prove from left to right, the other implication is trivial. Let $f\in \bar{k}(C)$ be such that
$$ div(f) = (P)-(Q) $$
so if $div(f) = 0$ then we are done. We have the vector space
$$ \mathscr{L}(Q) = \left\{ f\in \bar{k}(C) : div(f) \geq -(Q) \right\} \cup \left\{ 0 \right\} $$
which has dimension $\ell(Q) = dim_{\bar{k}} \mathscr{L}(Q)$. Since $\deg Q = 1$ and $g = 1$
we can use Corollary \ref{rrcor}c 
$$\ell(Q) = \deg Q - g + 1 = 1.$$
But since the constant functions are always in $\mathscr{L}(Q)$ we have by the dimension restriction that
they are the only ones, and $f \in \bar{k}$. This means that $div(f) = 0$ and we are done.
\end{proof}

\begin{thm} \label{isoteorem}
Let $E$ be an elliptic curve and $Pic^0(E) = Div^0(E)/Prin(E)$ be the Picard group, then
 $$ sum: Pic^0(E) \rightarrow E(\bar{k}) $$
is an isomorphism of abelian groups.
\end{thm}
\begin{proof}
 We begin by showing that there is a unique point $P \in E(\bar{k})$ associated to
each $D \in Div^0(E)$ as follows
$$ D \sim (P) -(O).$$
This will be given by a map
$$ \sigma: Div^0(E) \rightarrow E(\bar{k}). $$
From Corollary \ref{rrcor} we have that $\ell(D+(O)) = \deg(D+(O)) = 1$ since $\deg D = 0$.
Let then $f \in \bar{k}(E)$ be a generator for $\mathscr{L}(D+(O))$, so by definition
$$ div(f) \geq -D-(O). $$
But since $\deg(div(f)) = 0$ and $\deg(-D-(O)) = -1$ we have for some $P \in E(\bar{k})$ that
$$ div(f) = -D-(O)+(P) $$
which is exactly the definition of
$$ D \sim (P) - (O). $$

This point $P$ is unique, because if we assume that $P'$ is another point with the same
property, then
$$ (P) \sim D + (O) \sim (P') $$
so by Proposition \ref{3.3} we have that $ P = P'$.

The map $\sigma$ is easily seen to be a surjection, because for any $P \in E(\bar{k})$ we have
$$ \sigma((P)-(O)) = P.$$

Now if we can show that the kernel of $\sigma$ is exactly the principal divisors we are done.
Let us assume that $\sigma(D) = O$ so from definition we have that $D \sim (O)-(O) \sim (O)$
meaning $D - (O) = div(f)$ for some $f \in \bar{k}(E)$, so $D = div(f)$ is principal.
For the other implication we assume that $D = div(f)$ is principal. Using the definition and
letting $P$ be any point and $f, f' \in \bar{k}(E)$ a calculation yields
$$ \sigma(D) = \sigma(div(f)) = (P)-(O) $$
$$ div(f) \sim (P) - (O) $$
$$ div(f) - (P) - (O) = div(f') $$
$$ (O) - (P) = div(f') - div(f) = div(f' f) $$
So $ (P) \sim (O) $ which implies $P = O$ from Proposition \ref{3.3}.

For a proof that the group law on $E$ described earlier is the same as the group law on $Pic^0(C)$ I refer
to \cite{AEC}.

We have thus established the group isomorphism which we again will denote by
$$ \sigma : Div^0(E)/Prin(E) = Pic^0(E) \rightarrow E(\bar{k}).$$
\end{proof}

\begin{mydef}
 An \emph{isogeny} between two elliptic curves $E_1$ and $E_2$ is a morphism $\phi: E_1 \rightarrow E_2$
which satisfies $\phi(O) = O$. In addition, two curves are said to be \emph{isogenous} (of degree $n$) 
if there exists a non-zero isogeny (of degree $n$) between them.
\end{mydef}

The following is a nice result about isogenies using separability, see \cite{AEC} for a proof.
\begin{prop}\label{tri}
Let $\phi$ be a separable isogeny as below and $\ker \phi \subseteq \ker \psi$,
then there exists a unique $\lambda: E_2 \rightarrow E_3$ such that the diagram
commutes
$$
\xymatrix {
  E_1 \ar[r]^\phi \ar[dr]^\psi & E_2 \ar@{.>}[d]^\lambda \\
  & E_3 \\
}
$$
\end{prop}


Now note that $\phi: E_1 \rightarrow E_2$ induces a map on Picard groups
$$\phi^*: Pic^0(E_2) \rightarrow Pic^0(E_1).$$
In addition we have isomorphisms $E_1 \overset{\sigma_1}{\simeq} Pic^0(E_1)$ and 
$E_2 \overset{\sigma_2}{\simeq} Pic^0(E_2)$ from Theorem \ref{isoteorem}, combining this gives a composition
$$ E_2 \overset{\sigma_2^{-1}}{\rightarrow} Pic^0(E_2) \overset{\phi^*}{\rightarrow} Pic^0(E_1)
\overset{\sigma_1^{-1}}{\rightarrow} E_1. $$

\begin{prop}
 The composition above is the \emph{dual isogeny} of $\phi$, denoted $\widehat{\phi}: E_2 \rightarrow E_1$.
It has the following properties 
\begin{enumerate}
 \item $\widehat{\phi}\phi = [\deg\phi]$ on $E_1$ and $\phi\widehat{\phi} = [\deg\phi]$ on $E_2$.
 \item If $\psi: E_2 \rightarrow E_3$ is another isogeny then $\widehat{\psi \phi} = \widehat{\phi}\widehat{\psi}.$
 \item If $\lambda: E_1 \rightarrow E_2$ is another isogeny then $\widehat{\phi+\phi} = \widehat{\phi}+\widehat{\psi}.$
 \item $\deg\widehat{\phi} = \deg\phi.$
 \item $\widehat{\widehat{\phi}} = \phi.$
\end{enumerate}
\end{prop}

We end this section by a short comment on the dual isogeny. If $\phi$ is a separable map then
the first property follows by the following argument: letting $\deg\phi = m$ we have
that $\#\ker\phi = m$ which is clearly a subgroup of elements of order $m$
$$ \ker \phi \subseteq \ker [m].$$
Using Proposition \ref{tri} we have that there exists a unique $\lambda: E_2 \rightarrow E_1$
such that $\lambda\phi = [m]$ which by setting $\lambda =\widehat{\phi}$ is exactly what we want.
The case where $\phi$ is assumed inseparable can be found in \cite{AEC}.
%\section{Weil pairing and Tate module}

\begin{prop}
 The \emph{multiplication by $n$} map
$$ [n] : E \rightarrow E $$
$$ P \mapsto nP $$
has order $n^2$.
\end{prop}
\begin{proof}
The shortest proof relies heavily on the dual isogeny, so letting $d = deg [n]$ and using the properties of 
the dual isogeny we calculate
$$ [d] = \widehat{[n]}[n] = [n][n] = [n^2] $$
and since $End(E)$ is torsion free \cite{AEC} we get that $d = n^2$.
\end{proof}

A subgroup of $E(k)$ that will be of special interest to us is the group of points $P$
with finite order $n$, this is by definition the kernel of the multiplication by $n$ map.
\begin{mydef}
 The $n$-torsion subgroup denoted $E[n]$ is the group of points of order $n$ in $E$.
$$ E[n] = \{ P\in E : nP = O \} $$
\end{mydef}

We are now ready to construct a bilinear pairing between the $n$-torsion subgroups of
an elliptic curve and the roots of unity $\mu_n$. This will prove useful to us in coming
proofs. In addition it has well established applications within number theory, cryptography
and indentity based encryption.

The pairing we want to construct is of the form
$$ e_n : E[n] \times E[n] \rightarrow \mu_n $$
Let $T\in E[n]$ be an $n$-torsion point. From \cite{Lawrence} we know that there exists
$f \in \bar{k}(E)$ such that $div(f) = n(T) - n(O)$. Now letting $T' \in E[n^2]$ be such
that $nT' = T$, we have a function $g \in \bar{k}(E)$ such that
$$ div(g) = \sum_{R\in E[n]} (T'+R)-(R) $$
This follows from the fact that there are $n^2$ points in $E[n]$, the points $(R)$ in the
sum cancel, so we are left with $n^2 T' = nT = O$. Clearly $deg(div(g)) = 0$.

If we now form the composition $f \circ [n]$, we notice that the points $P = T' + R$ with
$R\in E[n]$ are those with the property $nP = T$. Now since $f$ has a root at $T$ from
construction, we see that $f \circ [n]$ has a root at $P$. Using the fact that $ord_P$ is a valuation
so that $div(g^n) = n\;div(g)$, and writing out the divisors of our functions we see that
$$ div(f \circ [n]) = n\sum_{R\in E[n]} (T'+R) - n\sum_{R\in E[n]} (R) = div(g^n) $$

Since our two rational functions $f \circ [n]$ and $g^n$ have the same divisors, they have the
same poles and zeros. Therefor they differ by multiplication of a constant, so
$f \circ [n] = \lambda g^n$ with $\lambda \in \bar{k}$. With a suitable choice of $\lambda$
we can assume that 
$$f \circ [n] = g^n$$
Letting $S \in E[n]$ be another $n$-torsion point and $X \in E(\bar{k})$ a point on the curve we calculate that
$$ g(X + S)^n = (f \circ [n])(X + S) = f([n]X + [n]S) = f([n]X) = g(X)^n $$
\begin{mydef}
 Given the above calculation the \emph{Weil pairing} is defined as
$$ e_n : E[n] \times E[n] \rightarrow \mu_n$$
$$ (S,T) \mapsto \frac{g(X + S)}{g(X)} $$
\end{mydef}

\begin{prop}
 The Weil pairing $e_n$ satisfies the following properties
\begin{enumerate}
 \item Bilinear in both variables. $e_n(P_1 + P_2, Q) = e_n(P_1,Q)e_n(P_2,Q)$ and similarly for the other variable.
 \item Alternating: $e_n(P,Q) = e_n(Q,P)^{-1}$.
 \item Non-degenerate: If $e_n(P,Q) = 1$ for all $P \in E[n]$ then $Q=O$.
 \item Galois invariant: For all $\sigma \in Gal(\bar{k}/k)$ we have $e_n(P,Q)^\sigma = e_n(\sigma(P),\sigma(Q))$.
\end{enumerate}
\end{prop}
\begin{proof}
 Bevis 1, 3 og 4.
\end{proof}

\begin{prop}
We have the following isomorphism of abelian groups
 $$ E[n] \simeq \mathbb{Z}/m\mathbb{Z} \times \mathbb{Z}/m\mathbb{Z} $$
\end{prop}
\begin{proof}
 fund. teorem. osv.
\end{proof}

The last proposition enables us to view automorphism of $E[n]$ as $2\times 2$ invertible matrices,
so we obtain a  mod $\ell$ galois representation
$$ Gal(\bar{k}/k) \overset{\rho}{\rightarrow} Aut(E[n]) \simeq GL_2(\mathbb{Z}/m\mathbb{Z}) $$
To avoid working with congruences and instead work with equalities, we can construct
and work with a field of characteristic 0. This is done by taking the inverse limit 
as introduced in chapter \ref{p-adic} of the sequence
$$ \dots \overset{[l]}{\rightarrow} E[\ell^{n+1}] \overset{[l]}{\rightarrow} E[\ell^{n}] \overset{[l]}{\rightarrow} E[\ell^{n-1}] \rightarrow \ldots $$
$$ T_\ell(E) = \varprojlim E[\ell^n] $$
This is called the \emph{$\ell$-adic Tate module} of E. Notice that since each the groups $E[\ell^n]$ has a
$\mathbb{Z}/\ell^n\mathbb{Z}$-module structure, $T_\ell(E)$ will have natrual structure as a module
over the ring og $\ell$-adic integers $\mathbb{Z}_\ell$.

Similarly we can in a sense ``glue'' together the Weil pairings
$$ e_{\ell^n} : E[\ell^n] \times E[\ell^n] \rightarrow \mu_{\ell^n} $$
by constructing the $\ell$-adic roots of unity, and we obtain what is called the
\emph{$\ell$-adic Weil pairing}.
$$ e: T_\ell(E) \times T_\ell(E) \rightarrow T_\ell(\mu) $$

%\section{Frobnius and finite fields} \label{frob}
Throughout this section our fields $k$ will be finite, so let $char(k) = p$ for
a prime $p$. This means that $k = \mathbb{F}_{q}$ for some $q = p^r$.

\begin{mydef}
 The \emph{Frobenius} endomorphism is the $p^{th}$-power map
$$ \phi: k \rightarrow k $$
$$ x \mapsto x^p $$
which induces a map on curves as follows
$$ \phi: E(k) \rightarrow E^{\phi}(k) $$
$$ (x_0,\ldots , x_n) \mapsto (x_0^p, \ldots , x_n^p) $$
where $E^{\phi}$ is the curve $E$ with $\phi$ applied to its coefficients.
$$E: y^2 = x^3 + ax + b \quad E: y^2 = x^3 + \phi(a)x + \phi(b) $$
\end{mydef}

We can apply the Frobenius endomorphism $r$ times $$\phi^r(x) = x^{p^r} = x^q$$
And since every finite field of $q$ elements is the splitting field of $x^{q}-x$, it is in other words
the fixed points of the $q^{th}$ Frobenius endomorphism
$$ \phi^r(x) = x \iff x \in \mathbb{F}_q $$
The same is true for all intermediate fields of size $p^k$ with $0 < k \leq r$, so in general
we have that the $\phi^k$ fixes the elements of the field $\mathbb{F}_{p^k}$.
\begin{prop}
 The degree map
$$ deg: Hom(E_1, E_2) \rightarrow \mathbb{Z} $$
is a positive quadratic form.
\end{prop}
\begin{proof}
 Clearly $deg(f) = deg(-f)$. The only thing that takes a proof is the
bilinearity of the pairing
$$ End(E_1, E_2) \times End(E_1, E_2) \rightarrow \mathbb{Z}$$
$$ (\phi, \psi) \mapsto deg(\phi + \psi) - deg(\phi) - deg(\psi) $$
For this proof we will make extentive use of the dual isogeny, but first
notice that we have an injection of multiplication by $n$ maps:
$$ [\quad]: \mathbb{Z} \rightarrow End(E_1) $$
A calculation then yields 
\begin{eqnarray*} 
 [\langle \phi,\psi \rangle] &=& [deg(\phi+\psi)]-[deg(\phi)]-[deg(\psi)] \nonumber \\
               &=& (\widehat{\phi+\psi})(\phi+\psi) - \widehat{\phi}\phi - \widehat{\psi}\psi \nonumber \\
	       &=& \widehat{\phi}\psi + \widehat{\psi}\phi
\end{eqnarray*}
The pairing is then shown to be linear in the first varible, the second variable is
similar.
\begin{eqnarray*}
 [\langle \phi_1+\phi_2, \psi \rangle] &=& \widehat{\psi}(\phi_1+\phi_2) + (\widehat{\phi_1+\phi_2})\psi \nonumber \\
			 &=& (\widehat{\psi}\phi_1+\widehat{\phi_1}\psi) + (\widehat{\psi}\phi_2 + \widehat{\phi_2}\psi) \nonumber \\
			 &=& [\langle \phi_1,\psi \rangle] + [\rangle \phi_2,\psi \rangle] 
\end{eqnarray*}
\end{proof}

\begin{thm} \label{frobkernel}
 Let $\phi$ be the $q^{th}$ frobenius map on $E/\mathbb{F}_q$. Then the map $1-\phi$ is seperable, and
$\#ker(1-\phi) = deg(1-\phi)$.
\end{thm}
\begin{proof}
  Recall from chapter \ref{diffsep} that a map $\psi$ is separable if and only if $\psi^*(\omega) \neq 0$,
where $\omega$ is the invariant differential. Using that the Frobenius $\phi$ is inseparable \cite{AEC}
we compute
\begin{eqnarray}
 (1-\phi)^*(\omega) &=& [1]^*\omega - \phi^*(\omega) \nonumber \\
		    &=& \omega - 0 \nonumber \\
		    &=& \omega \nonumber
\end{eqnarray}
thus $(1-\phi)^*(\omega) = 0$ if and only if $\omega = 0$, but the invariant differential is non-zero
so $(1-\phi)^*(\omega) \neq 0$ which means $1-\phi$ is separable.
\end{proof}

\begin{lemma}
 \textbf{(Cauchy-Schwartz inequality)}. Let $A$ be an abelian group and
$$ d: A \rightarrow \mathbb{Z} $$
a positive definite quadratic form. Then for all $\psi, \phi \in A$ the following holds
$$ |d(\psi-\phi)-d(\phi)-d(\psi)| \leq 2 \sqrt{d(\phi)d(\psi)} $$
\end{lemma}
\begin{proof}
 Let $\psi, \phi \in A$. From the definition of a quadratic form there is a bilinear pairing
$$ L(\psi, \phi) = d(\psi-\phi) - d(\psi) - d(\phi) $$
Using this definition, the fact that $d$ is positive definite and letting $m,n \in \mathbb{Z}$ where
$m = -L(\psi, \phi)$ and $n = 2d(\psi)$ we calculate

\begin{eqnarray}
 0 \leq d(m\psi - n\phi) &=& d(m\psi) + L(m\psi, n\phi) + d(n\phi) \nonumber \\
			 &=& m^2 d(\psi) + mnL(\psi,\phi) + n^2 d(\phi) \nonumber \\
			 &=& d(\psi) \left( 4d(\psi)d(\phi)-L(\psi, \phi)^2 \right) \nonumber 
\end{eqnarray}

where on the last line we make the substitution. If $d(\psi)=0$ the inequality is trivial, if
$d(\psi) \neq 0$ then we divide it out and obtain our result
$$L(\psi, \phi)^2 \leq 4d(\psi)d(\phi) $$
\end{proof}

\begin{thm}
 \textbf{(Hasse's theorem)}. Let $E$ be an elliptic curve over a finite field $k$ with $q$ elements, then
$$ |\#E(k) - q - 1| \leq 2\sqrt{q} $$
\end{thm}
\begin{proof}
 We let $\phi_q: E \rightarrow E$ be the $q^{th}$ Frobenius endomorphism on $E$ given by 
$(x,y) \mapsto (x^q, y^q)$. Recall that $\phi_q$ fixes our field of $q$ elements, thus
$$ P \in E(k) \quad \iff \quad \phi_q(P) = P$$
Writing out the right hand side of the implication we see that
$$ 0 = P - \phi_q(P) = (1 - \phi_q)(P) $$
which enables us to count the number of points in $E(k)$ by counting the number of points in the kernel
of the seperable map $1-\phi_q$. Recall from before that the number of points in the kernel is equal
to the degree of the seperable map
$$ \#E(k) = \# ker(1-\phi_q) = deg(1-\phi_q) $$
We have shown in that the degree map on $End(E)$ is a positive definite quadratic form, so
by using the inequality from the previous theorem we calculate
$$|deg(1-\phi_q) - deg(\phi_q) - deg(1)| = |\#E(k) - q - 1| \leq 2\sqrt{deg(\phi_q)} = 2\sqrt{q}$$
\end{proof}

\begin{prop} 
 If $\psi \in End(E)$ then $det(\psi_\ell) = deg(\psi)$, where $\psi_\ell$ is a $2\times2$ matrix acting
on the Tate module $T_\ell(E)$.
\label{detdeg}
\end{prop}
\begin{proof}
 We fix a basis $v_1,v_2 \in \mathbb{Z}_\ell \times \mathbb{Z}_\ell$ for $T_\ell(E)$ and denote the matrix
associated to this basis by
$$ \psi_\ell = \begin{pmatrix} a & b \\ c & d \end{pmatrix} $$
We now calculate by relying heavily on the $\ell$-adic Weil pairing,
$e: T_\ell(E) \times T_\ell(E) \rightarrow T_\ell(\mu)$.
\begin{eqnarray}
 e(v_1, v_2)^{deg(\psi)} &=& e([deg \psi]v_1, v_2) \nonumber \\
			 &=& e(\psi_\ell \widehat{\psi_\ell} v_1, v_2) \nonumber \\
			 &=& e(\psi_\ell v_1, \psi_\ell v_2) \nonumber \\
			 &=& e(a v_1 + c v_2, b v_1 + d v_2) \nonumber \\
			 &=& e(a v_1, d v_2) e(c v_2, b v_1) \nonumber \\
			 &=& e(a v_1, d v_2) e(b v_1, c v_2)^{-1} \nonumber \\
			 &=& e(v_1, v_2)^{ad} e(v_1, v_2)^{-bc} \nonumber \\
			 &=& e(v_1, v_2)^{ad - bc} \nonumber \\
			 &=& e(v_1, v_2)^{det \psi_\ell} \nonumber
\end{eqnarray}
Since the pairing is non-degenerate we obtain $deg(\psi) = det(\psi_\ell)$.
\end{proof}

Writing out the determinant of $1-A$ for any matrix $A$ we get
$$ \begin{vmatrix} 1-a & -b \\ -c & 1-d \end{vmatrix} = 1-(a+d)+ad-bc = 1-tr(A)+det(A) $$
so we see that $tr(\psi_\ell) = 1 + det(\psi_\ell) - det(1-\psi_\ell)$. Using the previous theorem we
get $$tr(\psi_\ell) = 1 + deg(\psi_\ell) - deg(1-\psi_\ell)$$ by substituting with the $q^{th}$ 
Frobenius endomorphism on $T_\ell(E)$ and setting $\tau = tr(\phi_q)$ we get
$$\#E(k) = 1 + q - \tau$$
where we know from Hasse's theorem that $|\tau| \leq 2\sqrt{q}$.

The next proposition will be used in chapter \ref{satoh}, it is easy to prove and gives a nice
expression of the Frobenius trace in terms of the dual isogeny.

\begin{prop}
 Let $\phi: E \rightarrow E$ be the $q^{th}$ Frobenius endomorphism and $\widehat{\phi}$ its dual, then
the following holds
$$ t = tr(\phi) = \phi + \widehat{\phi}$$
\end{prop}
\begin{proof}
 Recall that $1-\phi$ is seperable, so $$(1-\phi)(\widehat{1-\phi}) = deg(1-\phi) = \#ker(1-\phi) = \#E(k)$$
Expanding the product on the left we get
\begin{eqnarray}
 (1-\phi)(\widehat{1-\phi}) &=& (1-\phi)(1-\widehat{\phi}) \nonumber \\
			    &=& 1 - (\phi + \widehat{\phi}) + \phi\widehat{\phi} \nonumber \\
			    &=& 1 - (\phi + \widehat{\phi}) + q \nonumber
\end{eqnarray}
From before we had that $\#E(k) = q + 1 - t$ and we just calculated that $\#E(k) = q + 1 - (\phi +\widehat{\phi})$ so
the result follows.
\end{proof}




\section{Schoof's algorithm}
\subsection{Division polynomials}
Recall for this section that an elliptic curve corresponds to a lattice $\Lambda$
so we have an isomorphism
$$ \bar{k}/\Lambda \simeq E(\bar{k}) $$
$$ z \mapsto (\wp(z), \wp '(z)) $$
where $\wp(z)$ is the elliptic Weierstrass function. 

\begin{mydef}
 The \emph{division polynomials} are polynomials $\Psi_n(x,y) \in \mathbb{Z}[x,y,A,B]$
defined by the recurrence relations
\begin{align*}
  \Psi_0 &= 0 \\
  \Psi_1 &= 1 \\
  \Psi_2 &= 2y \\
  \Psi_3 &= 3x^4 + 6Ax^2 + 12Bx - A^2 \\
  \Psi_{2n+1} &= \Psi_{n+2} \Psi_n^3 - \Psi_{n+1}^3 \Psi_{n-1} \\
  \Psi_{2n}   &= (2y)^{-1} \Psi_n(\Psi_{n+2} \Psi_{n-1}^2 - \Psi_{n-2} \Psi_{n+1}^2)
\end{align*}
where $\Psi_n(x,y) = 0$ is and only if $(x,y) \in E[n]$.
\end{mydef}
The construction of these polynomials can be done in at least two ways and I will discuss
both of them briefly.

One way of doing this is to construct
a function having poles at the $n$-torsion points of our elliptic curve as follows
$$ f_n(z) = n^2 \prod(\wp(z) - \wp(u)) $$
where the product is taken over all $n$-torsion points of $\bar{k}/\Lambda$, denoted
$\bar{k}/\Lambda[n]$. This function has roots at exactly the $n$-torsion points by definition,
which is at least what we want. A more throrough examination of this method can be found
in [serge lang-ref].

Another way which is more elementary by highly computational is to work explicitly
with the addition formulas for elliptic curves. + mer forklaring.

Replacing the terms $y^2$ in $\Psi_n$ by $x^3 + Ax + B$ we obtain polynomials $\Psi_n '$ in
$\mathbb{F}_q[x]$ if is $n$ is odd or $y \mathbb{F}_q[x]$ if $n$ is even. To avoid
this distinction we define
$$
f_n(x) = \begin{cases}
          \Psi_n '(x,y) & \text{if n is odd} \\
	  \Psi_n '(x,y)/y & \text{if n is even}
         \end{cases}
$$


\begin{prop}
 Let $n \geq 2$ and $\Psi_n$ the division polynomial as defined above, then
$$ nP = (x - \frac{\Psi_{n-1} \Psi_{n+1}}{\Psi_n^2}, \frac{\Psi_{n+2} \Psi_{n-1}^2 - \Psi_{n-2} \Psi_{n+1}^2}{4y \Psi_n^3} )$$
\end{prop}

\subsection{Computing the number of points}
For an elliptic curve over $\mathbb{F}_q$ given by
$$ E: y^2 = x^3 + Ax + B $$
we want to compute the size of $\#E(\mathbb{F}_q)$, we know from before that
$$ \#E(\mathbb{F}_q) = q + 1 - t $$
where $t$ is the trace of the Frobenius as seen in section [referanse]. We know
that $t$ satisfies the Hasse bound namely
$$ |\#E(\mathbb{F}_q)-q-1|=|t| < 2\sqrt{q} $$
Let $S = \{3, 5, 7, 11, \ldots \, L \}$ be the set of odd primes $\leq L$ such
that the product is bigger than the Hasse interval
$$ N = \prod_{\ell \in S} \ell  > 4\sqrt{q} $$
If we can then calculate $t\, (mod \ell)$ for all $\ell \in S$ we can uniquely
determine $t\,(mod N)$ by invoking the Chinese remainder theorem,
which then by the Hasse bound is our Frobenius trace $t$.

The argument above is the gist of Schoof's algorithm, we will now look at
how to calculate $t\, (mod \ell)$. Let $\phi$ be the Frobenius endomorphism
resticted to $E[\ell]$ and let $q_\ell$, $\tau$ be $q$ and $t$ reduced modulo $\ell$
respectively. The computation of $\tau$ can then be done by checking if
$$ \phi^2(P) + q_\ell P = \tau \phi(P) $$
holds for $P \in E[\ell]$. To perform the addition on the left hand side of the
equality we need to distinguish the cases where the points are on a vertical line or not.
In other words we have to verify if for $P = (x,y) \in E[\ell]$ the following holds
$$ \phi^2 (P) = \pm q_\ell P $$
Noting that $-P = (x, -y)$ we write out the equality for the $x$-coordinates in terms of
division polynomials
$$ x^{q^2} = x - \frac{\Psi_{q_\ell-1} \Psi_{q_\ell+1}}{\Psi_{q_\ell}^2}(x,y) $$
Writing this out in terms of $f_n(x)$ and noting that for $n$ even we have
$\Psi_n(x,y) = y f_n(x)$, a calculation for $q_\ell$ even yields
\begin{eqnarray*}
 x^{q^2} &=& \frac{f_{q_\ell-1}(x) f_{q_\ell+1}(x)}{(f_{q_\ell} y)^2} \nonumber \\
	 &=& \frac{f_{q_\ell-1}(x) f_{q_\ell+1}(x)}{f_{q_\ell}^2 (x^3+Ax+B)} \nonumber \\
\end{eqnarray*}
The calculation for $q_\ell$ odd is similar and we get the equality

$$
x^{q^2} = \begin{cases}
           x - \frac{f_{q_\ell-1}(x) f_{q_\ell+1}(x)}{f_{q_\ell}^2 (x^3+Ax+B)} & \text{if } q_\ell \text{ is even} \\
	   x - \frac{f_{q_\ell-1}(x) f_{q_\ell+1}(x) (x^3+Ax+B)}{f_{q_\ell}^2(x)} & \text{if } q_\ell \text{ is odd} 
          \end{cases}
$$
We thus get two equations and we want to verify they have any solutions $P \in E[\ell]$. For
doing this we compute the following greatest common divisors
$$ gcd((x^{q^2} - x)f_{q_\ell}^2 (x^3+Ax+B)+f_{q_\ell-1}(x) f_{q_\ell+1}(x), f_\ell(x)) \quad (q_\ell \text{ even)}$$
$$ gcd((x^{q^2} - x)f_{q_\ell}^2(x)+f_{q_\ell-1}(x) f_{q_\ell+1}(x) (x^3+Ax+B), f_\ell(x)) \quad (q_\ell \text{ odd)}$$
We are now going to treat the rest in two cases, depending on the value from the above gcds.

\textbf{Case 1:} $gcd \neq 1$ meaning there exist a non-zero $\ell$-torsion point $P$ such that $\phi^2(P) = \pm q_\ell P$.
If $\phi^2 (P) = -q_\ell P$ we have that $\tau \phi(P) = 0$ but since $\phi(P) \neq 0$ we know that $\tau = 0$.
If $\phi^2(P) = q_\ell P$ we have that 
$$ 2 q_\ell P = \tau \phi(P) \Leftrightarrow \phi(P) = \frac{2 q_\ell}{\tau} $$
Substituting the last equality into $\phi^2(P) = q_\ell P$ we obtain
$$ \frac{4 q_\ell^2}{\tau^2} = q_\ell P \Leftrightarrow 4 q_\ell P = \tau^2 P $$
We thus obtain the congruence $\tau^2 \equiv 4q_\ell \, (mod \ell)$

\textbf{Case 2:} $gcd = 1$ so $\phi^2(P) \neq \pm q_\ell P$ meaning they are not equal or on the same
vertical line for any $\ell$-torsion point $P$. This enables us to do the addition
$\phi^2(P) + q_\ell P$ using the appropriate addition formulas.

\subsection{Modular polynomials}
fixme
\section{Satoh's algorithm}
fixme


\end{document}
